\documentclass[12pt]{article}

\usepackage[margin=1in]{geometry}
\usepackage{hyperref}
\usepackage{amsmath}
\usepackage{amssymb}
\usepackage{bm}
\usepackage{graphicx}

\def\ket#1{|#1\rangle}
\def\bra#1{\langle#1}
\def\sech{{\,\rm sech}}
\def\etothe#1{{\rm e}^{#1}}
\def\a{{\bf a}}
\def\M{{\bf M}}
% \def\Real{{\rm Re}}\def\Imag{{\rm Im}}

\begin{document}

\section{Introduction}

The multidimensional stability of plasma electron holes has recently been
extensively studied assuming that the most unstable perturbation is a
parallel shift of the hole position []. The eigenvalue, which
determines the complex frequency $\omega$, has previously been
evaluated using a ``Rayleigh Quotient'' which provides an
approximation accurate to second order in any deviations from the
exact unstable mode structure. The results are in reasonable agreement
with PIC simulations, but some discrepancies have been noted. The
purpose of the present work is to discover whether more accurate mode
shape determination, including deviations from the shift mode, can
explain those discrepancies, and to quantify by analysis how important
the deviations are.

Two main phenomena have been observed in transverse instability
simulations that are not represented by the shift mode analysis. They
are (1) narrowing of the unstable mode structure relative to the shift
mode, when near marginal oscillatory stability; (2) generation of long
parallel length streaks or waves on the whistler branch, during
oscillatory instability at high magnetic field.

A relatively compact approach to managing a generalization of the
unstable mode of a Vlasov-Poisson problem previously advocated[] is to
represent the perturbation of potential in terms of the eigenmodes
(which are orthogonal) of a judiciously chosen differential
operator. In the present context, the operator generally used
represents the Poisson equation for steady (or very slowly varying)
potential ($\nabla^2\phi-n=0$ in appropriately normalized units). The
particle density difference $n_0(z)$ associated with the potential
equilibrium $\phi_0(z)$, can be determined from this equation. Then
for infinitessimally slow linearized potential perturbations $\phi_1$
about this equilibrium the perturbed density is
$n_1=\phi_1dn_0/d\phi_0$, and the resulting Poisson equation can be
written using the operator $V_a \equiv(\nabla^2-dn_0/d\phi_0)$ acting
on $\phi_1$ as $V_a\phi_1=0$. This $V_a$ is called the ``adiabatic''
operator, and the associated density perturbation $n_1$ the adiabatic
density (perturbation). For low frequencies, expansion of the
potential perturbation is most naturally in terms of the eigenmodes of
$V_a$. For purely growing instabilities $\omega=0+i\omega_i$, at the
threshold $\omega_i\sim 0$, evidently the adiabatic response is nearly
equal to the total because changes are infinitessimally slow; so the
non-adiabatic part $\tilde V$ of the operator is small compared with
$V_a$ in the perturbed Poisson equation. Consequently to lowest order
the perturbation unstable mode is equal to the eigenmode of $V_a$ with
zero eigenvalue. For a solitary potential structure in a uniform
background, such a zero eigenvalue always exists, and its eigenmode
has the form of a uniform shift of the equilibrium.

By the preceding argument, determining where $\omega_i=0$, i.e. the
stability threshold, can be accomplished exactly using just the shift
mode, provided that the real part $\omega_r$ of the mode frequency is
zero. However, some hole instabilities are oscillatory:
$\omega_r\not=0$, $\omega_i>0$. Then the marginal unstable mode is not
purely the shift mode, and the extent to which it includes
contributions from other eigenmodes of $V_a$ becomes an important
question. It can be explored by carrying the perturbation analysis to
first order in the other eigenmodes, in much the way that
time-independent perturbation theory is used in quantum
mechanics. This approach was successfully pursued in an early study of
the one-dimensional instability of a train of electron holes that
leads eventually to hole merger[]. However, the few subsequent efforts
to apply it [] have been of limited utility either because of
expanding about the wrong eigenmode, or, more fundamentally because of
adopting inappropriate approximations to the solution of the Vlasov
equation, which constitutes the complementary (and more difficult)
part of the Poisson-Vlasov system: the non-adiabatic perturbation. The
present work instead solves the Vlasov problem by numerical
integration internally over the prior orbit. Moreover it avoids any
perturbative approximation for the relative amplitudes of the different
modes, instead showing how the eigenmode continuum contribution can be
represented through a single amplitude corresponding to the
external wave dispersion relation.


\section{Eigenmode Expansion}
Let us adopt a minimalist bra-ket notation for the eigenmodes in which
we expand: $\ket{e}$, where the label $e$ being either real
$p,q,\dots$, or integer, $j,l,\dots$, will denote respectively
continuum or discrete eigenmodes. The inner product of any two
Hilbert-space vectors (complex potential functions of $z$) denotes an
overlap integral over (parallel) spatial coordinate
$\bra{u}\ket{s}\equiv \int u^*(z)s(z)dz$. Insofar as the eigenmodes
are orthonormal, we write $\bra{u}\ket{s}=\delta_{us}$, where for
continuum modes such that $u$ and $s$ are real parameters, this is
(approximately) a Dirac delta function, $\delta_{us}=\delta(u-s)$,
whereas for discrete modes $\delta_{jl}$ is the Kronecker delta.

\subsection{Adiabatic Response Eigenmodes}
In scaled units the one-dimensional Poisson equation, assuming
immobile uniform ion density, is $d^2\phi/dz^2=n-1$. The equilibrium
analyzed is chosen to be of the form
$\phi_0=\psi\sech^4(z/4)$. Writing for brevity $S=\sech(z/4)$ and
$T=\tanh(z/4)$ and denoting the $z$-derivative by prime, the
linearized equilibrium (``adiabatic'') Poisson operator for this hole
potential can then be written
\begin{equation}
  \label{eq:1}
  V_a = \left[{d^2\over dz^2} - {dn\over d\phi_0}\right]
  = \left[{d^2\over dz^2} - {\phi_0^{'''}\over \phi_0'}\right]
  = \left[{d^2\over dz^2} +{30\over 16}S^2- 1\right].
\end{equation}
Notation $V$ reminds us that this is partly the Vlasov operator,
transforming a potential into a density, and the subscript $a$ denotes
adiabatic (meaning steady).  The eigenmodes of this
operator, satisfying $V_a\ket{u}=\lambda \ket{u}$, can be found [] by
applying the raising operator $4 {d\over dz}-T$ five times to
the function $ \etothe{uz/4}$ yielding
\begin{equation}
  \label{eq:2}
  \begin{split}
 \ket{u}(z)= &\exp(uz/4)\{[-15u^4 + (420S^2 - 225)u^2 - 945S^4 +
 840S^2 - 120]T \\
 &\qquad\qquad+ u[u^4 + (-105S^2 + 85)u^2 + (945S^4 - 1155S^2
 +274)]\}.
\end{split}
\end{equation}
For real $u$, as $u z\to+\infty$ the mode is bounded (and tends to
zero) only if $u$ is one of the discrete roots of the polynomial in
the braces obtained by letting $S\to 0$ and $T\to {\rm sign}(u)$. These
are $u=j=1,2,3,4,5$. The odd numbered discrete modes are symmetric in
$z$, and the even numbered are antisymmetric. In contrast, imaginary
$u$-values $u=ip$ give rise to the continuum modes which are formally
finite at infinity. Their overlap integral over a finite domain
exists; and, as the domain tends to infinity, it tends to a delta
function. The continuum eigenmodes have definite parity only if $u$
reverses sign with $z$; so we write $u=i \sigma_z p$ where
$\sigma_z={\rm sign}(z)$, and positive $p$ represents antisymmetric
outwardly-propagating waves.

The corresponding eigenvalues are
\begin{equation}
  \label{eq:3}
  \lambda_u= u^2/16-1=-p^2/16-1.
\end{equation}
Normalized so that
$\bra{j}\ket{l}=\delta_{jl}$ they are given in Table \ref{discrete}. 
\begin{table}[ht]
  \center
\begin{tabular}{ccc}\
  Mode & Eigenvalue & Normalized Form\\
  $\ket{1}$& -15/16&$S(21S^4 - 28S^2 + 8)\sqrt{30}/32$\\
  $\ket{2}$& -12/16&$TS^2(3S^2 - 2)\sqrt{105}/8$\\
  $\ket{3}$& -7/16 &$-S^3(9S^2 - 8)\sqrt{105}/32 $\\
  $\ket{4}$&  0 &$-3TS^4\sqrt{70}/16 $\\
  $\ket{5}$&  9/16 &$3S^5\sqrt{35}/32 $\\
\end{tabular}
\caption{Discrete eigenmodes.\label{discrete}}
\end{table}
\noindent
In particular $\lambda_4=0$ for the shift mode
$\ket{4}\propto d\phi_0/dz$, and it is the predominant
perturbation in essentially all linear instabilities. Moreover, it
couples only to antisymmetric modes; so $\ket{2}$ is the only other
discrete mode that needs to be considered.

It can
be shown that the the continuum modes are ``normalized'', in the sense
that then $\bra{p}\ket{q}=\delta_{pq}=\delta(p-q)$, by dividing eq.\
(\ref{eq:2}) by the factor
\begin{equation}
  [8\pi(p^2+1^2)(p^2+2^2)(p^2+3^2)(p^2+4^2)(p^2+5^2)]^{1/2}.
  \label{eq:4}
\end{equation}
Far
from the hole ($z\gg1$) the normalized oscillatory continuum modes are
sinusoidal with amplitude $1/\sqrt{8\pi}=0.19947$, and parallel wavenumber
$|k_\parallel|=p/4$. The eigenmodes are plotted in Fig.\
\ref{modeplots}.
\begin{figure}\center
  \includegraphics[width=0.6\hsize]{continmodez}
  \caption{Eigenmodes of the adiabatic response operator.\label{modeplots}}
\end{figure}

\subsection{Including Non-adiabatic Response}
We now must consider the full linearized Poisson equation including the
non-adiabatic response arising from the solution to the time-dependent
Vlasov equation, $\tilde{V}$, as well as the adiabatic response
$V_a$. The form of the non-adiabatic Vlasov operator $\tilde{V}$ will
be discussed later. For a perturbation
$\phi_1(y,z,t)=\hat\phi(z){\rm e}^{i(k_\perp y-\omega t)}$ with
transverse wavenumber $k_\perp$ and frequency $\omega$, Poisson's
equation becomes
\begin{equation}
  \label{eq:5}
  (-k_\perp^2+V_a+\tilde{V})\ket{\hat\phi}=0,
\end{equation}
in which it is convenient to regard $k_\perp^2\equiv\lambda_\perp$ as the
full eigenvalue. 
We suppose its solution can be expanded as a sum of scalar
amplitudes $a_u$ times the eigenmodes of $V_a$, $\ket{u}$:
\begin{equation}
  \label{eq:6}
  \ket{\hat\phi}=\sum_u a_u \ket{u},
\end{equation}
where the summation notation also includes an integral over the
continuum modes.

Then we can invoke the orthogonal properties of the
adiabatic eigenmodes and form the inner product
\begin{equation}
  \label{eq:7}
  \bra{s}|(%-16k_\perp^2+
  V_a+\tilde{V})\ket{\hat\phi}
  = %-16k_\perp^2+
  \lambda_s\bra{s}\ket{s}a_s+\sum_u\bra{s}|\tilde{V}\ket{u}a_u,
\end{equation}
which must be equal to $\lambda_\perp\bra{s}\ket{s}a_s$ to satisfy eq.\ \ref{eq:5}.
In particular, choosing the predominant mode $\bra{4}|$ for
$\bra{s}|$, for which $\lambda_4=0$, we get an eigenvalue equation
$0=(\lambda_4-\lambda_\perp)\bra{4}\ket{4}a_4+\sum_u\bra{4}|\tilde{V}\ket{u}a_u$.

If all the amplitudes $a_u$ for $u\not=4$ are negligible, then
$\tilde{V}$ contributes a correction ($\lambda_{4}^{(1)}$) to the eigenvalue
\begin{equation}
  \label{eq:8}
  \lambda_{4}^{(1)}= {\bra{4}|\tilde{V}\ket{4}\over\bra{4}\ket{4}}=\lambda_\perp-\lambda_4
\end{equation}
The expression $\bra{4}|\tilde{V}\ket{4}/\bra{4}\ket{4}$ is the ``Rayleigh
quotient'' approximation for the eigenvalue of $V_a+\tilde{V}$
(because $\lambda_4=0$). It is
physically the (normalized) jetting force on particles because of the
unperturbed electric field $-d\phi_0/dz\propto\bra{4}|$, acting on the
non-adiabatic density perturbation $\tilde n = \tilde{V}\ket{4}$, integrated
over the entire hole. It balances the (normalized) Maxwell shear
stress from the transverse kinking of the hole ($k_\perp^2$) to
make the total force zero.

Actually $\tilde{V}$ is a complicated nonlinear function of the complex
frequency $\omega$ of the mode; and for specified $k_\perp$ the
dispersion relation between $\omega$ and $k_\perp$ must be solved by
some kind of iterative procedure searching for an $\omega$ that
satisfies eq.\ (\ref{eq:8}). The imaginary part of $\omega$ thus found
determines the stability of the hole. This approximation has been
extensively pursued in recent publications[] and yielded stability results
that are in reasonable (but not perfect) agreement with simulation.
The question at hand is whether analysis can determine approximately
the magnitude of the other coefficients $a_u$ for $u\not=4$, and
therefore give a more accurate perturbation structure $\ket{\hat\phi}$
and $\omega$.

If instead of the approximation (\ref{eq:8}), we are able to evaluate all the
matrix coefficients $\bra{s}|\tilde{V}\ket{u}$ then in principle we can regard eq.\
(\ref{eq:7}) instead of (\ref{eq:8}), as a matrix eigen-system we must
solve to find the $\omega$ that permits a non-zero solution for the
vector $a_u$. The off-diagonal matrix entries $s\not=u$ are the
coupling of the potential modes by the non-adiabatic Vlasov operator
$\tilde{V}$.  The condition for the existence of a solution is that the
determinant of the matrix
$[(-\lambda_\perp+\lambda_s)\bra{s}\ket{s}\delta_{su}+\bra{s}|\tilde{V}\ket{u}]$ should
be zero. A possible iterative scheme might consist of finding its
eigenvalues, and adjusting $\omega$ until the eigenvalue for which
$a_4$ is predominant over the other coefficients becomes zero.

Such a program faces formidable practical challenges, however, because
each evaluation of $\tilde{V}\ket{u}$, requires a computation
involving multiple-dimension integrations over space and velocity
distribution --- repeated for each mode $\ket{u}$ and each adjustment
of $\omega$.  Moreover, in principle, the continuum contains
infinitely many modes, and the matrix contains the square of the
number of modes.  Obviously we require this number to be reduced. We
later show how the entire continuum contribution can be reduced to a
single amplitude. Also, continuum modes $\ket{p}$ extend to
$|z|=\infty$: an integration range that for computation needs to be
reduced. We will show how that is done.

Formally, any mode for which $\tilde{V}\ket{u}a_u$ is not negligible
must be retained in the sum of $\bra{s}|\tilde{V}\ket{u}$ in eq.\
\ref{eq:7}, giving for the dominant mode
\begin{equation}
  \label{eq:9}
  0=(\lambda_4-\lambda_\perp)\bra{4}\ket{4}a_4+\bra{4}|\tilde{V}\ket{4}a_4+\sum_{u\not=4}\bra{4}|\tilde{V}\ket{u}a_u,
\end{equation}
but also for $s\not=4$
\begin{equation}
  \label{eq:10}
 0=(\lambda_s-\lambda_\perp)\bra{s}\ket{s}a_s+\bra{s}|\tilde{V}\ket{4}a_4+\sum_{u\not=4}\bra{s}|\tilde{V}\ket{u}a_s.
\end{equation}


The well known approach of time-independent perturbation theory in
elementary quantum mechanics (e.g.[] Dirac Section 43) regards $\tilde{V}$ as
systematically small, and takes $a_s$ for $s\not=4$ also to be first
order small relative to $a_4$. The first order approximation of eq.\
\ref{eq:10} then drops the final sum term, giving
\begin{equation}
  \label{eq:11}
  a_s = {\bra{s}|\tilde{V}\ket{4}a_4\over (\lambda_\perp-\lambda_s)\bra{s}\ket{s}}.
\end{equation}
Substituting back (with $s\to u$) into eq.\ \ref{eq:9}, the
eigenvalue to second order is
\begin{equation}
  \label{eq:12}
 \lambda_\perp\simeq \lambda_4+\lambda_{4}^{(1)}+\lambda_{4}^{(2)}=\lambda_4
  +{\bra{4}|\tilde{V}\ket{4}\over \bra{4}\ket{4}}
    +\sum_{u\not=4}
    {\bra{4}|\tilde{V}\ket{u}\bra{u}|\tilde{V}\ket{4}\over
      (\lambda_\perp-\lambda_u)\bra{4}\ket{4}\bra{u}\ket{u}}.
\end{equation}
Normally the substitution $\lambda_\perp\simeq\lambda_4$ is made in
the final sum; but we do not need to do that since we consider
$\lambda_\perp$ to be given and $\omega$ to be changed to achieve
equality in this equation.

These perturbation equations (\ref{eq:11}) and (\ref{eq:12}) are
appropriate for coupled discrete modes $\ket{j}$ when the amplitudes
$a_u\bra{u}\ket{u}$ for $u\not=4$ are small compared with
$a_4\bra{4}\ket{4}$. Those equations give the first order mode
amplitude and second order eigenvalue correction from them.  However,
in our case, for the continuum modes, $\tilde{V}$ is not
systematically small everywhere, and an altered approach appropriate
to the actual form of $\tilde{V}$ must be adopted. We shall see that
in order to obtain converged integrals for the relevant continuum
inner products it is necessary to use the difference between
$\tilde V \ket{q}$ and a pure wave operator expression
$\tilde V_w\ket{q}$.


\section{The Eigenmode Coefficients}
\subsection{The Non-adiabatic Linearized Vlasov Operator}

The operator $\tilde{V}$ transforms a potential perturbation $\ket{u}$
into a non-adiabatic density perturbation $\tilde n$, both of which
are complex functions of $z$. It does so by solving the linearized
time-dependent Vlasov equation for the non-adiabatic distribution
function perturbation $\tilde f(z,v)$ and integrating it:
$\tilde n =\int \tilde f dv$. The solution can be found in terms
of an integral over past time along the linearized Vlasov equation's
``characteristic'', that is, the unperturbed orbit $\bm x(t)$:
\begin{equation}
  \label{eq:phim}
  \Phi(\bm x,\bm v,t)\equiv 
  \int_{-\infty}^t \phi_1(\bm x(\tau),t-\tau ) d\tau,
\end{equation}
where $\phi_1(\bm x,t)$ is the potential perturbation and
$\bm x(\tau)$ is the past position of the unperturbed orbit
that has velocity $\bm v$ at $(\bm x,t)$.  When a uniform magnetic
field in the $z$-direction is present with cyclotron frequency
$\Omega$, and we take the form of the potential perturbation to be
$\phi_1=\hat\phi(z){\rm e}^{i(k_\perp y-\omega t)}$ whose transverse
wavenumber is $k_\perp$, and the background perpendicular velocity
distribution to be Maxwellian, then the transverse perpendicular velocity
dependence can be expressed as a sum over cyclotron harmonics[].
The non-adiabatic parallel distribution function perturbation is then
\begin{equation}\label{eq:ftmagnetic}
    \tilde f(z,y,v,t) = {\rm e}^{i(k_\perp y-\omega t)}
    \sum_{m=-\infty}^\infty \tilde f_m(z,v)= {\rm e}^{i(k_\perp
      y-\omega t)} \sum_{m=-\infty}^\infty b_m \Phi_m,
\end{equation}
with
\begin{equation}
  \label{mweight}
   b_m=i\left[\omega_m
  {\partial f_{\parallel0}\over \partial W_\parallel}
  +m\Omega {f_{\parallel0}\over T_\perp}\right]
q_e{\rm e}^{-\zeta_t^2}I_m(\zeta_t^2),\qquad
\Phi_m=\int_{-\infty}^t \hat\phi(z(\tau))\etothe{-i\omega_m(\tau-t)}d\tau.
\end{equation}
Here $\zeta_t^2=k^2T_\perp/\Omega^2m_e$, $I_m$ is the modified Bessel
function, and $\omega_m=m\Omega+\omega$. The unperturbed equilibrium
parallel distribution function depends only on parallel energy $W_\parallel$
(not $z$ directly) and is written $f_{\parallel 0}$, but for the
perturbed parallel distributions we omit the $\parallel$ subscript for
brevity, and from now on omit the $y,t$ dependencies as implicitly
$ {\rm e}^{i(k_\perp y-\omega t)}$ (so in $\Phi_m$, the upper limit is
$t=0$). The prior integral $\Phi_m$ is a function of parallel position
$z$ and velocity $v$. The weight $b_m$ is independent of $z$ but
depends on $W_\parallel$ through the $f_{\parallel0}$ distribution.  We can
regard each harmonic $m$ as giving a perturbed density contribution
$\tilde n_m(z)=\int \tilde f_m dv$. When $\hat\phi$ is a sum over
modes $\ket{u}$,
\begin{equation}
  \label{eq:13}
\tilde  n_m=\tilde{V}_{m}\ket{\hat\phi}=\tilde{V}_{m}\sum_u a_u\ket{u}=\int \sum_u a_u
  \tilde f_{um} dv=\sum_u a_u \tilde n_{um},
\end{equation}
where $\tilde f_{um}$ denotes $\tilde f_m$ with
$\ket{\hat\phi}=\ket{u}$ substituted in eq. (\ref{eq:ftmagnetic}), and
similarly $\tilde n_{um}=\int \tilde f_{um}dv$. Then
$\tilde V \ket{\hat\phi}=\sum_m\tilde
V_m\ket{\hat\phi}=\sum_{u,m}a_u\tilde n_{um}$.

Solving analytically for $\Phi_m$ and $n_m$ in the non-uniform
equilibrium of an electron hole seems too difficult. The approach
taken here is to perform the required integrations numerically.
 Evaluating $\tilde
n_{um}$ is performed using the same code for each $m$, but using the
different $\omega_m$ in eq.\ (\ref{mweight}). The inner products
we need are $\bra{s}|\tilde V\ket{u}=\int_{-\infty}^\infty s^*(z)\tilde
n_{u}(z) dz$.

\subsection{The external continuum wave contribution}

In the region far outside the hole, $|z/4|\gg 1$ where
$T=\tanh(z/4)\to\pm1$ and $S=\sech(z/4)\to0$ in eq.\ (\ref{eq:2}), the
normalized continuum modes (taking them to be antisymmetrically
outward propagating) are
\begin{equation}
  \label{eq:14}
 \ket{p}=A\etothe{i\sigma_zpz/4}=\sigma_z{(-15p^4+225p^2-120) +ip (p^{4}-85p^{2}+274)\over
      [8\pi(p^2+1^2)(p^2+2^2)(p^2+3^2)(p^2+4^2)(p^2+5^2)]^{1/2}}\etothe{i\sigma_zpz/4},
\end{equation}
where $\sigma_z={\rm sign}(z)$. Thus they are purely sinusoidal
waves. For large enough $|z|$ the influence of the local hole
potential becomes negligible, and such waves should there be
identified with the normal modes of the uniform background plasma.
The applicable normal mode for the present electrostatic
approximation, in the (usually well justified) cold electrostatic limit,
ignoring ion response, has dispersion relation (including the upper
hybrid waves)
\begin{equation}
  \label{eq:15}
  k_\parallel^2/k_\perp^2={\omega^2(\Omega^2+\omega_{pe}^2-\omega^2)
    \over(\Omega^2-\omega^2)(\omega_{pe}^2-\omega^2)}.
\end{equation}
(In our normalized units $\omega_{pe}=1$.)  Assuming that $\omega$ and
$k_\perp$ are prescribed, there is only one $|k_\parallel|$ that
satisfies this dispersion relation; and corresponding to it, the
continuum eigenmode of $V_a$ has $k_\parallel=\sigma_zp/4$. The mode
number $p$ is taken positive, but $k_\parallel$ is signed. In this
subsection we analyze just this single mode. The approach will be
mathematically justified in the following subsection.

For a single continuum mode, one can immediately calculate the value
of $\Phi_m(z)$ for external positions and \emph{inward} orbit velocity
(e.g.\ negative $v$ on the positive-$z$ side) using $z'=z(\tau)$ and
$\tau=-(z-z')/v$, as
\begin{equation}
  \label{Phiwave}
  \Phi_{m}^{wave}(z) = \int_{\sigma_z\infty}^zA{\rm e}^{i[\sigma_zpz'/4+\omega_m
    (z-z')/v]}dz'/v
  ={A{\rm e}^{i\sigma_zpz/4} \over i(\sigma_zpv/4-\omega_m)}={A{\rm e}^{ik_\parallel z} \over
    i(k_\parallel v-\omega_m)},
\end{equation}
where dropping the infinity limit is justified by positive imaginary
part of $\omega$.  This $\Phi_{m}^{wave}$ for the external 
region applies for all $z$ down to where the eigenmode begins to
deviate significantly, because of non-zero hole potential, from the
external wave; obviously that $z$ is of order a few times 4.

The non-adiabatic distribution perturbation for \emph{outward}
(positive $v$ at positive $z$) orbit velocity, by contrast, is
strongly affected by the rapid variation of potential across the hole
at $z\sim 0$ whose effect is carried into the external region by the
particle orbits. However, for large enough $|z|$ the effects of the
hole and earlier parts of the orbit become negligible because of
dephasing between oscillatory contributions from different
velocities. The dephasing effect attenuates the influence on density
in a distance of several times $z_l\sim 1/\omega_m$, which is finite
but (for $m=0$ at least) typically exceeds the hole extent itself
($\sim 4$) because $\omega$ is small.  Thus the very distant
($|z|\gg z_l$) $\tilde n$ perturbations for inward and outward going
velocities are given approximately by integrating $dv$ the same
$\Phi_m^{wave}$ expression for $z\gg z_l$, but with $v$ of opposite
sign.

The total wave density perturbation can then be considered to be a
constant $\tilde V_m^{wave}$ times $\ket{p}$, where
\begin{equation}
  \label{vmwave}
  \tilde V_m^{wave}= \int_{-\infty}^\infty { b_m(v) \over
    i(k_\parallel v-\omega_m)} dv.
\end{equation}
If $f_{\parallel0}$ is an unshifted Maxwellian, and only $m=0$ is
included, the total wave operator
$\tilde V_w\equiv \sum_m\tilde V_m^{wave}$ is proportional to the
plasma dispersion function $Z$, and in the small $k_\parallel/\omega$
limit becomes $\tilde V_w=1+(k_\parallel/\omega)^2$.  However, that
approximation effectively implements the limit
$\Omega\gg \omega_{pe}$, and gives a dispersion relation
$k_\parallel^2/k_\perp^2 = \omega^2/(\omega_{pe}^2-\omega^2)$, rather
than the full cold plasma expression for finite $\Omega/\omega$: eq.\
(\ref{eq:15}). The full expression arises when the $|m|=1$ terms in
the harmonic sum for the kinetic electrostatic dispersion
relation\footnote{See, e.g., Swanson (1989) section 4.4.}, to lowest
order in $\zeta_t^2$, are also included. The resulting analytic form
is
\begin{equation}
  \label{eq:24}
  \tilde V_w= 1 + \left(k_\parallel\over\omega\right)^2 +
  k_\perp^2 {T_\perp/T_\parallel\over\omega^2-\Omega^2}.
\end{equation}

The crucial point is that $(\tilde{V}-\tilde{V}_{w})\ket{q}$ is
localized, tending to zero in the region beyond $z_l$, which means
that overlap integrals of the form
$\bra{p}|\tilde{V}-\tilde{V}_{w}\ket{q}$ exist finite over an infinite
domain. An important secondary feature is that for external positions
$|z|\ge z_d$ the inward velocity part of $\tilde V_m^{wave}$,
consisting of
\begin{equation}
  \label{inward}
  \tilde V_m^{in}= \int_{0}^\infty { b_m(-\sigma_x|v|) \over
    i(-|k_\parallel v|-\omega_m)}\sigma_x d|v|,
\end{equation}
is \emph{exactly} equal to the actual non-adiabatic density
perturbation that accounts for the hole's presence and the full
eigenmode structure. So the cancellation of the part of
$\tilde V-\tilde V_w$ for inward orbits only is exact
$(\tilde V^{in}-\tilde V_m^{in})\ket{q}=0$. Therefore the non-zero
contribution to external region $z$-integrals of
$(\tilde V-\tilde V_w)\ket{q}$ arises only from outward orbits
$(\tilde V^{out}-\tilde V_m^{out})\ket{q}\not=0$. The necessary
integral of outward orbits must be carried out to an upper limit for
which $|z|\gg z_l$. We shall consider explicitly only parallel
distributions $f_{\parallel0}$ that are symmetric in $v$. In that case
$b_m$ is symmetric, and the overlap integrals can be calculated for a
single sign of $v$ and then doubled to give the total.


\subsection{Reducing the continuum to give the dispersion matrix}

Now we discuss mathematically how the previous considerations allow us
to reduce the continuum contribution to effectively a single mode.
Using $s,u\to q,p$ etc to observe our notation for continuum modes, and
supposing them to be normalized ($\bra{q}\ket{p}=\delta_{qp}$), we
write eq.\ \ref{eq:10} explicitly as an integral $a(p)dp$ over an
amplitude distribution function $a(p)$ rather than a sum.
Proceeding with no assumption about the size of cross coupling of secondary
modes, the eigenmode equation inner product with mode $\bra{s}|$ gives
the replacement for \ref{eq:10} as
\begin{equation}
  \label{eigengen}
  0=(\lambda_s-\lambda_\perp)\bra{s}\ket{s}a_s+\sum_j\bra{s}|\tilde{V}\ket{j}a_j
+\int\bra{s}|\tilde{V}\ket{p}a(p)dp.
\end{equation}
We apply this equation for the three modes $s=4,2,q$. First for the
continuum mode ($\bra{s}|=\bra{q}|$) with the adiabatic terms moved to
the left hand side:
\begin{equation}
  \label{eigq}
  (-\lambda(q)+\lambda_\perp)a(q)=\sum_j\bra{q}|\tilde{V}\ket{j}a_j
+\int\bra{q}|\tilde V\ket{p} a(p)dp.
\end{equation}
We write the continuum integral using $\bra{q}|\tilde V_w\ket{p}=
\tilde V_w \delta_{qp}$as 
\begin{equation}
  \label{eq:18}
  \int \bra{q}|\tilde{V}\ket{p}a(p)dp=\int\bra{q}|\tilde{V}_{w}+(\tilde{V}-\tilde{V}_{w})\ket{p}a(p)dp
  =\tilde{V}_{w}a(q) +\int\bra{q}|(\tilde{V}-\tilde{V}_{w})\ket{p}a(p)dp.
\end{equation}
As can be seen in Fig. \ref{modeplots}, the form of the continuum
modes in the inner region is almost independent of $p$. Actually for
small $p$, which is our interest here, the differences are even
smaller than that figure shows. Therefore, to that degree of
approximation, we can replace $\ket{p}$ with $\ket{q}$ in the final
term, giving $\int\bra{q}|(\tilde{V}-\tilde{V}_{w})\ket{p}a(p)dp\simeq
\bra{q}|\tilde{V}-\tilde{V}_{w}\ket{q}\int a(p)dp$, and obtain the
equation
\begin{equation}
  \label{eigq}
  (-\lambda(q)+\lambda_\perp-\tilde V_w)a(q)=\sum_j\bra{q}|\tilde{V}\ket{j}a_j
+\bra{q}|\tilde V-\tilde V_w\ket{q}\int a(p)dp.
\end{equation}
Since the right hand side is a weak function of $q$, this equation
implies that $a(q)$ is a localized function of $q$ centered on the
$q$-value for which its coefficient on the left hand side is
approximately zero. We can write the coefficient using eq.\ (\ref{eq:24})as
\begin{equation}
  \label{}
  k_\perp^2+k_\parallel^2+1-\tilde
  V_w\simeq k_\perp^2\left(1-{T_\perp/T_\parallel\over\omega^2-\Omega^2}\right)
  +k_\parallel^2\left(1-{1\over\omega^2}\right)
= {1\over4^2}\left({1\over\omega^2}-1\right)\left(q_0^2-q^2\right)
\end{equation}
where
\begin{equation}
  \label{eq:26}
  q_0=4k_\perp\sqrt{1+{T_\perp/T_\parallel\over\Omega^2-\omega^2}}
  \Bigg/\sqrt{{1\over\omega^2}-1}.
\end{equation}
The coefficient $(-\lambda(q)+\lambda_\perp-V_w) $ is zero when
$q^2=q_0^2$, which is the approximate dispersion relation of the wave
in a uniform plasma. It is equal to the cold plasma expression
(\ref{eq:15}) when $T_\perp/T_\parallel=1$.



We integrate eq.\ (\ref{eigq})  $dq/(\lambda_\perp-\lambda(q)-V_w)$, recognizing again the approximate independence of $q$ of
the right hand side to find
\begin{equation}
  \label{eigqint}
  \int a(q) dq =\int {dq\over \lambda_\perp-\lambda(q)-V_w}
  \left(\sum_j\bra{q}|\tilde{V}\ket{j}a_j
+\bra{q}|\tilde V-\tilde V_w\ket{q}\int a(p)dp \right).
\end{equation}

In view of the resonant form of the expression for $a(q)$, we can regard the
integral over this resonance $\int a(q) dq$ as quantifying the total
continuum perturbation.  The integral encounters the two poles of $a(q)$ at
$q=\pm q_0$.
% =\pm4k_\perp\omega/\sqrt{1-\omega^2}$.
Near them the real part of $(\lambda_\perp-\lambda(q)-V_w)$ becomes small. Its
imaginary part comes from the small, \emph{necessarily positive},
imaginary part of
$\omega=\omega_r+i\omega_i$. It can then quickly be shown that
$q_0$ has a small positive imaginary part: $q_0=q_{0r}+i\epsilon$.
\iftrue%
The infinite integral along the real
$q$-axis can be closed by returning in the upper part of the complex
plane ($\Im(q)$ positive) where the eigenmode tends
to zero. The resonant integral is then just $2\pi i$ times the residue
at the positive $q_0$ pole, which is the mathematical
  justification for our physical assumption of outwardly
  propagating waves. Writing it
$\int {dq\over \lambda_\perp-\lambda(q)-V_w}=1/K$,  we find
\begin{equation}
  \label{Kdef}
  K\simeq-{(1/\omega^2-1)q_0\over16\pi i}
  ={-k_\perp\over4\pi i}\sqrt{{1\over\omega^2}-1}
\sqrt{1+{T_\perp/T_\parallel\over\Omega^2-\omega^2}}.  
\end{equation}
Substituting elsewhere the
value $q=q_0$ at the integrand's pole, and introducing the shorthand
notation $\int a(p) dp= a_q$, eq.\ (\ref{eigqint}) becomes
\begin{equation}
  \label{eigqint2}
  (K-\bra{q_0}|\tilde V-\tilde V_w\ket{q_0})a_q %\int a(p)dp
  = \sum_j\bra{q_0}|\tilde{V}\ket{j}a_j.
\end{equation}

Using again the approximation that over the resonance
$\bra{j}|\tilde V\ket{p}$ can be regarded as independent of $p$, the
discrete modes then satisfy, for $l=4,2$,
\begin{equation}
  \label{eigenl}
  (\lambda_\perp-\lambda_l)a_l=\sum_j\bra{l}|\tilde{V}\ket{j}a_j
  +\int\bra{l}|\tilde{V}\ket{p}a(p)dp
  =\sum_j\bra{l}|\tilde{V}\ket{j}a_j
  +\bra{l}|\tilde{V}\ket{q_0}\int a(p)dp.
\end{equation}
We regard the amplitudes of the three modes 4,2,$q_0$ as composing a
column 3-vector whose transpose is
\begin{equation}
  \label{avector}
  \a^T=(a_4,a_2,a_q)=(a_4,a_2,\int a(p)dp)
\end{equation}
the three relations in \ref{eigqint2} and \ref{eigenl} can be
considered a matrix equation $\M\a=0$, where
\begin{equation}
  \label{aveq}
  \M=\left[
  \begin{array}{ccc}
    \lambda_4-\lambda_\perp+\bra{4}|\tilde V\ket{4}
    &\bra{4}|\tilde V\ket{2}&\bra{4}|\tilde V\ket{q_0}\\
    \bra{2}|\tilde V\ket{4}
    &\lambda_2-\lambda_\perp+\bra{2}|\tilde
      V\ket{2}&\bra{2}|\tilde V\ket{q_0}\\
    \bra{q_0}|\tilde V\ket{4}
    &\bra{q_0}|\tilde V\ket{2}&-K+\bra{q_0}|\tilde V-\tilde V_w\ket{q_0}
  \end{array}\right]
\iffalse
\left[
  \begin{array}{c}
    a_4\\
    a_2\\
    \int a(p)dp
  \end{array}\right]=0.
\fi.
\end{equation}
The complex determinant $||\M||$ must be zero for a non-trivial
solution, and if $\omega$ is iterated until $||\M||=0$, the value
found will be the corresponding mode frequency, and the mode structure
will be given by the solution of $\M\a=0$. Note\footnote{
The ordered perturbative approximations found in prior sections correspond,
in this formulation, to neglecting $\bra{2}|\tilde V\ket{q_0}$,
$\bra{q_0}|\tilde V\ket{2}$ and $\bra{2}|\tilde V\ket{2}$; then
determining $a_2/a_4$ from the second row, and $\int a(p)dp/a_4$ from the third
row.}


Our reduction of the resonant continuum mode uses just the $m=0,\pm 1$
cyclotron harmonics to determine $q_{0}.$ That is justified by
supposing that the finite Larmor radius parameter
$\zeta_t=k_\perp\sqrt{T_\perp/m_e}/\Omega$ is small, and recognizing
$I_m(\zeta_t^2)\propto \zeta_t^{2m}$. If $\zeta_t$ is not small (because
$\Omega\not\gg k_\perp v_{t\perp}$), instead, the dependence
$\etothe{-\zeta_t^2}$ predominates in eq.\ (\ref{eq:ftmagnetic}), and
harmonics up to $|m|\sim 3\zeta_t$ approximate a continuous integral
over the perpendicular Maxwellian, with the relevant range of
$\omega_m$ approximately $\omega\pm 3k_\perp v_{t\perp}$. The
resultant frequency spread exceeds $\sim\omega$ for the whistler mode
(for which $\omega < \Omega$) implying that high harmonic
(perpendicular Landau) damping is strong. The continuum mode
contribution itself is then suppressed, and it is reasonable to ignore
coupling to the continuum modes, reducing $\M$ to its upper left
2$\times$2 submatrix.

\section{Evaluation of Matrix Elements}
\subsection{Internal Numerical Evaluation of inner products}

The evaluation of the overlap integrals (forces) that appear in $\M$
for the inner hole region is carried out numerically using methods
that have been documented in detail previously [] for mode
$\ket{4}$. In summary the process consists, for each relevant orbit
energy $W$, of numerically integrating to obtain the relationship
between $z$ and the prior time $\tau$ for the unperturbed orbit, and
simultaneously accumulating the integral $\Phi_m(z)$, using a discrete
(but non-uniform) $z$-mesh.  Trapped orbits $W<0$ require a sum over
all prior bounces in the potential well, which is represented by a
multiplying bounce-resonant factor. Passing orbits use a single transit
across the domain $-z_d<z<z_d$, and for the discrete modes there is no
external contribution. The continuum mode at the incoming boundary
($z_{in}=-(v/|v|)z_d$), unlike the discrete modes, has a non-zero
value, which is provided by the wave expression
$\Phi_m(z_{in})=\Phi_m^{wave}$, eq.\ (\ref{Phiwave}).  Inside the
prior time integration loop the overlap integrals
$\int s^*(z)\tilde f_m(z)dz$ are simultaneously
accumulated. Afterwards they are integrated $dv$ over the relevant
range of parallel energy $W$, and summed over relevant harmonics $m$
to give the hole-region contributions to $\bra{s}|\tilde
V\ket{u}$. Except when $s$ is the continuum mode (the bottom row of
$\M$), these internal values provide the full evaluation of the
overlap integrals. For $s=q_0$, however, extra contributions arise
from the external region, which we now describe.

\subsection{External Analytic Integration}

In the external region $|z|>z_d$, the prior integral $\Phi$ giving
$\tilde f$ for discrete modes $\ket{j}$ is non-zero only for outward
moving orbits. We focus for definiteness on $z>z_d$ and positive $v$.
For inward moving orbits, by contrast, $\tilde f$ is zero externally
for the operations $\tilde V\ket{j}$ and
$(\tilde V -\tilde V_w)\ket{q_0}$. Consequently, the external
contribution to the forces $\bra{q_0}|\tilde V \ket{j}$ and
$\bra{q_0}|\tilde V -\tilde V_m\ket{q_0}$ arises from the external
integration $\int_{z_d}^\infty q_0^*(z) \tilde n dz$ only for \emph{outward} orbits.

To assist with understanding, figure \ref{externfig2} shows a case
illustrating the relevant
\begin{figure}[ht]
  \includegraphics[width=0.8\hsize]{extern3}
  \caption{Illustrating the process of calculating the inner products
    involving $\ket{q_0}$ and $\ket{4}$. Passing density contributions
    from positive, outward, velocity orbits integrated over all
    relevant energies. The square point locates the edge of the inner
    hole region, chosen as $z_d=30$. The scaled $\Phi$ shown is for the
    last included energy $W=6$.\label{externfig2}}
\end{figure}
density perturbations $\tilde n_q=\tilde V_0^{out}\ket{q}$,
$\tilde n_w=\tilde V_0^{out}\ket{q}$, and $\tilde n_4=\tilde V\ket{4}$
arising from positive velocity orbits. The upper panel shows the
relevant curves for the continuum mode in the inner and outer regions.
The densities $\tilde n_q$ and $\tilde n_w$ are very different in the
inner region $|z|<z_d$ and well outside it but they converge to each
other in the wave region $|z|\gg z_l$ ($z_l\sim 20$ for this
relatively high frequency illustration). Also shown is the curve of
$\tilde V_w\ket{q}/2$ which is the average of the inward and
outward densities. It has a shape similar to $\tilde n_w$ but is
somewhat smaller because $\tilde V_0^{in}< \tilde V_0^{out}$. The
relevant contribution comes from the difference between $\tilde n_q$
and $\tilde n_w$. 

The lower panel shows $\tilde n_4$ which is substantial in the inner
region and converges to zero for $|z|\gg z_l$, on it one can see
residual oscillations caused by discrete contributions at low velocity
approximating the $dv$ integral, which die out at large $z$. In the
upper panel there are also some oscillations but they are barely
visible. The ``$\Phi$ (scaled)'' curve illustrates (only) the shape of
the final contribution to the velocity integral from high velocity,
showing how long the wavelength of oscillations becomes there; its
contribution to $\tilde n$ is small because $f_{\parallel0}$ is
negligible at high $v$.


Now we explain how the external integrations are performed mostly
analytically.  In the external region, the velocity is constant; so
for $\ket{j}$, which has zero perturbed potential there, $\Phi$ is a
simple time delay factor times its value at the join between internal
and external regions $\Phi(z_d)$:
\begin{equation}
  \label{phi4}
  \Phi_{\ket{j}}(z)=\etothe{i\omega_m(z-z_d)/v}   \Phi_{\ket{j}}(z_d).
\end{equation}
The corresponding expression for the continuum mode, which has
non-zero external potential, may be
found by substituting the wave expression \ref{eq:14} for  $\ket{q_0}$
which can be integrated analytically to give an extra term, arriving
at 
\begin{equation}
  \label{phiq}
  \Phi_{\ket{q_0}}(z)=\etothe{i\omega_m(z-z_d)/v}   \Phi_{\ket{q_0}}(z_d)
  +{A\etothe{ik_\parallel z_d}\over i(k_\parallel v-\omega_m)}[\etothe{ik_\parallel(z-z_d)}
  -\etothe{i\omega_m(z-z_d)/v}].
\end{equation}
Recalling that the wave operator is simply constant, its external
contribution to $\Phi_{\ket{w}}$ for a particular positive velocity is
${A\etothe{ik_\parallel z}\over i(k_\parallel v-\omega_m)}$, which
exactly cancels the first term in the square bracket of
$\Phi_{\ket{q_0}}(z)$. Such cancellation is essential to produce a
finite value for $\bra{q_0}|\tilde V -\tilde V_w\ket{q_0}$. Hence,
when forming
$(\tilde V -\tilde V_w)\ket{q_0}$, the required external prior
integral is
\begin{equation}
  \label{phidiff}
  \Phi_{\ket{q_0-w}}(z)\equiv \Phi_{\ket{q_0}}(z)-\Phi_{\ket{w}}
  =\etothe{i\omega_m(z-z_d)/v}   \Phi_{\ket{q_0}}(z_d)
  %-\Phi_{w}(z_d)]
  -{A\etothe{ik_\parallel z_d}\over i(k_\parallel v-\omega_m)}
  \etothe{i\omega_m(z-z_d)/v}.
\end{equation}
To obtain the total external force we can carry out the inner product
$z$-integration analytically as
\begin{equation}
  \label{extforce}
  \int_{z_d}^\infty q_0^*(z) \Phi_u(z)  dz =
  {A^*\Phi_u(z_d)\etothe{-ik_\parallel z_d}\over i(k_\parallel  v-\omega_m)}
  + {\delta_{q_0u}|A|^2v\over (k_\parallel v-\omega_m)^2},
\end{equation}
where $\Phi_u$ refers to the ``ket'' mode (j or $q_0-w$), and the
$|A|^2$ term is present only if we are constructing
$\bra{q_0}|\tilde V -\tilde V_w)\ket{q_0}$, as indicated by
$\delta_{q_0u}$. This force quantity is multiplied by the
weighting factor $b_m$ (equation \ref{mweight}) and integrated
(numerically) over parallel velocity so as to produce the total inner
product. For asymmetric $f_{\parallel0}$ the process would need to be
carried out also for negative velocity and $z$, but since we take
$f_{\parallel0}$ to be symmetric the integration is carried out only
for positive $v$ and $z$; the result is then doubled to account for
force exerted at negative $z$. The sum over relevant cyclotron
harmonics $m$ is performed last.

\end{document}

%%% Local Variables:
%%% mode: latex
%%% TeX-master: t
%%% End:
