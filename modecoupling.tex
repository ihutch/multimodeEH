\documentclass[12pt]{article}

\usepackage[margin=1in]{geometry}
\usepackage{hyperref}
\usepackage{amsmath}
\usepackage{amssymb}
\usepackage{graphicx}

\def\ket#1{|#1\rangle}
\def\bra#1{\langle#1}
\def\sech{{\,\rm sech}}
\def\etothe#1{{\rm e}^{#1}}
\def\a{{\bf a}}
\def\M{{\bf M}}
% \def\Real{{\rm Re}}\def\Imag{{\rm Im}}

\begin{document}

\section{Introduction}

The stability of plasma electron holes has recently been extensively studied
assuming that the most unstable perturbation is a parallel shift of
the hole position []. The eigenvalue, which determines the complex
frequency $\omega$, has previously been evaluated using a ``Rayleigh
Quotient'' which provides an approximation accurate to second order in
any deviations from the exact unstable mode structure. The results are
in reasonable agreement with PIC simulations, but some discrepancies
have been noted. The purpose of the present work is to discover
whether more accurate mode shape determination, including deviations
from the shift mode, can explain those discrepancies, and to quantify
by analysis how important the deviations are.

Two main phenomena have been observed in transverse instability
simulations that are not represented by the shift mode analysis. They
are (1) narrowing of the unstable mode structure relative to the shift
mode, when near marginal oscillatory stability; (2) generation of long
parallel length streaks or waves on the whistler branch, during
oscillatory instability at high magnetic field.

A relatively compact approach to managing a generalization of the
unstable mode of a Vlasov-Poisson problem [] is to represent the
perturbation of potential in terms of the eigenmodes (which are
orthogonal) of a judiciously chosen differential operator. In the
present context, the operator generally used represents the Poisson
equation for steady (or very slowly varying) potential
($\nabla^2\phi-n=0$ in appropriately normalized units). The particle
density $n_0(z)$ associated with the potential equilibrium
$\phi_0(z)$, can be determined from this equation. Then for
infinitessimally slow linearized potential perturbations $\phi_1$
about this equilibrium the perturbed density is
$n_1=\phi_1dn_0/d\phi_0$, and the resulting Poisson equation can be
written as the operator $V_a =(\nabla^2-dn_0/d\phi_0)$ acting on
$\phi_1$. This $V_a$ is called the ``adiabatic'' operator, and the
associated density perturbation $n_1$ the adiabatic density
(perturbation). The expansion of the potential perturbation is most
naturally in terms of the eigenmodes of $V_a$. For purely growing
instabilities $\omega=0+i\omega_i$, at the threshold $\omega_i\sim 0$,
evidently the adiabatic response is nearly equal to the total because
changes are infinitessimally slow; so the non-adiabatic part
$\tilde V$ of the operator is small compared with $V_a$ in the
perturbed Poisson equation. Consequently to lowest order the
perturbation unstable mode is equal to the eigenmode of $V_a$ with
zero eigenvalue. For a solitary potential structure in a uniform
background, such a zero eigenvalue always exists, and its eigenmode
has the form of a uniform shift of the equilibrium.

By the preceding argument, determining where $\omega_i=0$, i.e. the
stability threshold, can be accomplished exactly using just the shift
mode, provided that the real part $\omega_r$ of the mode frequency is
zero. However, some hole instabilities are oscillatory:
$\omega_r\not=0$, $\omega_i>0$. Then the marginal unstable mode is not
purely the shift mode, and the extent to which it includes
contributions from other eigenmodes of $V_a$ becomes an important
question. It can be explored by carrying the perturbation analysis to
first order in the other eigenmodes, in much the way that
time-independent perturbation theory is used in quantum
mechanics. This was successfully pursued in an early study of the
one-dimensional instability of a train of electron holes, which leads
eventually to hole merger[]. However, the few subsequent efforts to
apply it [] have been of limited utility either because of perturbing
about the wrong eigenmode, or, more fundamentally because of adopting
inappropriate approximations to the solution of the Vlasov equation,
which constitutes the complementary (and more difficult) part of the
Poisson-Vlasov system: the non-adiabatic perturbation. The present
work instead solves the Vlasov problem by numerical integration over
the prior orbit.


\section{Eigenmode Expansion}
Let us adopt a minimalist bra-ket notation for the eigenmodes in which
we expand: $\ket{e}$, where the label $e$ being either real
$p,q,\dots$, or integer, $j,l,\dots$, will denote respectively
continuum or discrete eigenmodes. The inner product of any two
Hilbert-space vectors (complex potential functions of $z$) denotes an
overlap integral over (parallel) spatial coordinate
$\bra{u}\ket{s}\equiv \int u^*(z)s(z)dz$. Insofar as the eigenmodes
are orthonormal, we write $\bra{u}\ket{s}=\delta_{us}$, where for
continuum modes such that $u$ and $s$ are real parameters, this is
(approximately) a Dirac delta function, $\delta_{us}=\delta(u-s)$,
whereas for discrete modes $\delta_{jl}$ is the Kronecker delta.

\subsection{Adiabatic Response Eigenmodes}
In scaled units Poisson's equation is $d^2\phi/dz^2=n-1$. The
equilibrium analyzed is chosen to be of the form
$\phi_0=\psi\sech^4(z/4)$. Writing for brevity
$S=\sech(z/4)$ and $T=\tanh(z/4)$ and denoting the $z$-derivative by
prime, the linearized equilibrium (``adiabatic'')
Poisson operator for this hole potential can then be written
\begin{equation}
  \label{eq:1}
  V_a = \left[{d^2\over dz^2} - {dn\over d\phi_0}\right]
  = \left[{d^2\over dz^2} - {\phi_0^{'''}\over \phi_0'}\right]
  = \left[{d^2\over dz^2} +{30\over 16}S^2- 1\right].
\end{equation}
Notation $V$ reminds us that this is partly the Vlasov operator,
transforming a potential into a density, and the subscript $a$ denotes
adiabatic (meaning steady).  The eigenmodes of this
operator, satisfying $V_a\ket{u}=\lambda \ket{u}$, can be found [] by
applying the raising operator $4 {d\over dz}-T$ five times to
the function $ \etothe{uz/4}$ yielding
\begin{equation}
  \label{eq:2}
  \begin{split}
 \ket{u}(z)= &\exp(uz/4)\{[-15u^4 + (420S^2 - 225)u^2 - 945S^4 +
 840S^2 - 120]T \\
 &\qquad\qquad u[u^4 + (-105S^2 + 85)u^2 + (945S^4 - 1155S^2
 +274)]\}.
\end{split}
\end{equation}
For real $u$, as $u z\to+\infty$ the mode is bounded (and tends to
zero) only if $u$ is one of the discrete roots of the polynomial in
the braces obtained by letting $S\to 0$ and $T\to {\rm sign}(u)$. These
are $u=j=1,2,3,4,5$. The odd numbered discrete modes are symmetric in
$z$, and the even numbered are antisymmetric. In contrast, imaginary
$u$-values $u=ip$ give rise to the continuum modes which are formally
finite at infinity. Their overlap integral over a finite domain
exists; and, as the domain tends to infinity, it tends to a delta
function. The continuum eigenmodes have definite parity only if $u$
reverses sign with $z$; so we write $u=i \sigma_z p$ where
$\sigma_z={\rm sign}(z)$, and positive $p$ represents antisymmetric
outwardly-propagating waves.

The corresponding eigenvalues are
\begin{equation}
  \label{eq:3}
  \lambda_u= u^2/16-1=-p^2/16-1.
\end{equation}
Normalized so that
$\bra{j}\ket{l}=\delta_{jl}$ they are given in Table \ref{discrete}. 
\begin{table}[ht]
  \center
\begin{tabular}{ccc}\
  Mode & Eigenvalue & Normalized Form\\
  $\ket{1}$& -15/16&$S(21S^4 - 28S^2 + 8)\sqrt{30}/32$\\
  $\ket{2}$& -12/16&$TS^2(3S^2 - 2)\sqrt{105}/8$\\
  $\ket{3}$& -7/16 &$-S^3(9S^2 - 8)\sqrt{105}/32 $\\
  $\ket{4}$&  0 &$-3TS^4\sqrt{70}/16 $\\
  $\ket{5}$&  9/16 &$3S^5\sqrt{35}/32 $\\
\end{tabular}
\caption{Discrete eigenmodes.\label{discrete}}
\end{table}
\noindent
In particular $\lambda_4=0$ for the shift mode
$\ket{4}\propto d\phi_0/dz$, and it is the predominant
perturbation in essentially all linear instabilities. Moreover, it
couples only to antisymmetric modes; so $\ket{2}$ is the only other
discrete mode that needs to be considered.

It can
be shown that the the continuum modes are ``normalized'', in the sense
that then $\bra{p}\ket{q}=\delta_{pq}=\delta(p-q)$, by dividing eq.\
(\ref{eq:2}) by the factor
\begin{equation}
  [8\pi(p^2+1^2)(p^2+2^2)(p^2+3^2)(p^2+4^2)(p^2+5^2)]^{1/2}.
  \label{eq:4}
\end{equation}
Far
from the hole ($z\gg1$) the normalized oscillatory continuum modes are
sinusoidal with amplitude $1/\sqrt{8\pi}=0.19947$, and parallel wavenumber
$k_\parallel=p/4$. The eigenmodes are plotted in Fig.\
\ref{modeplots}.
\begin{figure}\center
  \includegraphics[width=0.6\hsize]{continmodez}
\caption{Eigenmodes of the adiabatic response operator.\label{modeplots}}
\end{figure}

\subsection{Including Non-adiabatic Response}
We now must consider the full linearized Poisson equation including the
non-adiabatic response arising from the solution to the time-dependent
Vlasov equation, $\tilde{V}$, as well as the adiabatic response
in $V_a$. The form of the non-adiabatic Vlasov operator $\tilde{V}$ will
be discussed later. For a perturbation
$\phi_1(y,z,t)=\hat\phi(z){\rm e}^{i(k_\perp y-\omega t)}$ with
transverse wavenumber $k_\perp$ and frequency $\omega$, Poisson's
equation becomes
\begin{equation}
  \label{eq:5}
  (-k_\perp^2+V_a+\tilde{V})\ket{\hat\phi}=0,
\end{equation}
in which it is convenient to regard $k_\perp^2\equiv\lambda_\perp$ as the
full eigenvalue. 
We suppose its solution can be expanded as a sum of scalar
amplitudes $a_u$ times the eigenmodes of $V_a$, $\ket{u}$:
\begin{equation}
  \label{eq:6}
  \ket{\hat\phi}=\sum_u a_u \ket{u},
\end{equation}
where the summation notation also includes an integral over the
continuum modes.

Then we can invoke the orthogonal properties of the
adiabatic eigenmodes and form the inner product
\begin{equation}
  \label{eq:7}
  \bra{s}|(%-16k_\perp^2+
  V_a+\tilde{V})\ket{\hat\phi}
  = %-16k_\perp^2+
  \lambda_s\bra{s}\ket{s}a_s+\sum_u\bra{s}|\tilde{V}\ket{u}a_u,
\end{equation}
which must be equal to $\lambda_\perp\bra{s}\ket{s}a_s$ to satisfy eq.\ \ref{eq:5}.
In particular, choosing the predominant mode $\bra{4}|$ for
$\bra{s}|$, for which $\lambda_4=0$, we get an eigenvalue equation
$0=(\lambda_4-\lambda_\perp)\bra{4}\ket{4}a_4+\sum_u\bra{4}|\tilde{V}\ket{u}a_u$.

If all the amplitudes $a_u$ for $u\not=4$ are negligible, then
$\tilde{V}$ contributes a correction ($\lambda_{4}^{(1)}$) to the eigenvalue
\begin{equation}
  \label{eq:8}
  \lambda_{4}^{(1)}= {\bra{4}|\tilde{V}\ket{4}\over\bra{4}\ket{4}}=\lambda_\perp-\lambda_4
\end{equation}
The expression $\bra{4}|\tilde{V}\ket{4}/\bra{4}\ket{4}$ is the ``Rayleigh
quotient'' approximation for the eigenvalue of $V_a+\tilde{V}$
(because $\lambda_4=0$). It is
physically the (normalized) jetting force on particles because of the
unperturbed electric field $-d\phi_0/dz\propto\bra{4}|$, acting on the
non-adiabatic density perturbation $\tilde n = \tilde{V}\ket{4}$, integrated
over the entire hole. It balances the (normalized) Maxwell shear
stress from the transverse kinking of the hole ($k_\perp^2$) to
make the total force zero.

Actually $\tilde{V}$ is a complicated nonlinear function of the complex
frequency $\omega$ of the mode; and for specified $k_\perp$ the
dispersion relation between $\omega$ and $k_\perp$ must be solved by
some kind of iterative procedure searching for an $\omega$ that
satisfies eq.\ (\ref{eq:8}). The imaginary part of $\omega$ thus found
determines the stability of the hole. This approximation has been
extensively pursued in recent publications[] and yielded stability results
that are in reasonable (but not perfect) agreement with simulation.
The question at hand is whether analysis can determine approximately
the magnitude of the other coefficients $a_u$ for $u\not=4$, and
therefore give a more accurate perturbation structure $\ket{\hat\phi}$
and $\omega$.

If instead of the approximation (\ref{eq:8}), we are able to evaluate all the
matrix coefficients $\bra{s}|\tilde{V}\ket{u}$ then in principle we can regard eq.\
(\ref{eq:7}) instead of (\ref{eq:8}), as a matrix eigen-system we must
solve to find the $\omega$ that permits a non-zero solution for the
vector $a_u$. The off-diagonal matrix entries $s\not=u$ are the
coupling of the potential modes by the non-adiabatic Vlasov operator
$\tilde{V}$.  The condition for the existence of a solution is that the
determinant of the matrix
$[(-\lambda_\perp+\lambda_s)\bra{s}\ket{s}\delta_{su}+\bra{s}|\tilde{V}\ket{u}]$ should
be zero. A possible iterative scheme might consist of finding its
eigenvalues, and adjusting $\omega$ until the eigenvalue for which
$a_4$ is predominant over the other coefficients becomes zero.

Such a program faces formidable practical challenges, however, because
each evaluation of $\tilde{V}\ket{u}$, requires a computation involving
multiple-dimension integrations over space and velocity distribution
--- repeated for each mode $\ket{u}$ and each adjustment of $\omega$.
Moreover, in principle, the continuum contains infinitely many modes,
and the matrix contains the square of the number of modes.  Obviously
we require this number to be reduced! Also, continuum modes $\ket{p}$
extend to $|z|=\infty$: an integration range that for computation
needs to be reduced.

Formally, any mode for which $\tilde{V}\ket{u}a_u$ is not negligible
must be retained in the sum of $\bra{s}|\tilde{V}\ket{u}$ in eq.\
\ref{eq:7}, giving for the dominant mode
\begin{equation}
  \label{eq:9}
  0=(\lambda_4-\lambda_\perp)\bra{4}\ket{4}a_4+\bra{4}|\tilde{V}\ket{4}a_4+\sum_{u\not=4}\bra{4}|\tilde{V}\ket{u}a_u,
\end{equation}
but also for $s\not=4$
\begin{equation}
  \label{eq:10}
 0=(\lambda_s-\lambda_\perp)\bra{s}\ket{s}a_s+\bra{s}|\tilde{V}\ket{4}a_4+\sum_{u\not=4}\bra{s}|\tilde{V}\ket{u}a_s.
\end{equation}


The well known approach of time-independent perturbation theory in
elementary quantum mechanics (e.g.[] Dirac Section 43) regards $\tilde{V}$ as
systematically small, and takes $a_s$ for $s\not=4$ also to be first
order small relative to $a_4$. The first order approximation of eq.\
\ref{eq:10} then drops the final sum term, giving
\begin{equation}
  \label{eq:11}
  a_s = {\bra{s}|\tilde{V}\ket{4}a_4\over (\lambda_\perp-\lambda_s)\bra{s}\ket{s}}.
\end{equation}
Substituting back (with $s\to u$) into eq.\ \ref{eq:9}, the
eigenvalue to second order is
\begin{equation}
  \label{eq:12}
 \lambda_\perp\simeq \lambda_4+\lambda_{4}^{(1)}+\lambda_{4}^{(2)}=\lambda_4
  +{\bra{4}|\tilde{V}\ket{4}\over \bra{4}\ket{4}}
    +\sum_{u\not=4}
    {\bra{4}|\tilde{V}\ket{u}\bra{u}|\tilde{V}\ket{4}\over
      (\lambda_\perp-\lambda_u)\bra{4}\ket{4}\bra{u}\ket{u}}.
\end{equation}
Normally the substitution $\lambda_\perp\simeq\lambda_4$ is made in
the final sum; but we do not need to do that since we consider
$\lambda_\perp$ to be given and $\omega$ to be changed to achieve
equality in this equation.

Equations (\ref{eq:11}) and (\ref{eq:12}) apply fully for coupled
discrete modes $\ket{j}$ when the amplitudes $a_u\bra{u}\ket{u}$
for $u\not=4$ are small compared with $a_4\bra{4}\ket{4}$. Those
equations give the first order mode amplitude and second order
eigenvalue correction from them.  However, in our case, for the
continuum modes, $\tilde{V}$ is not systematically small everywhere,
and an altered approach appropriate to the actual form of $\tilde{V}$
must be adopted.


\section{Calculating the Eigenmode coefficients}
\subsection{The Non-adiabatic Linearized Vlasov Operator}

The operator $\tilde{V}$ transforms a potential perturbation $\ket{u}$
into a non-adiabatic density perturbation $\tilde n$, both of which
are complex functions of $z$. It does so by solving the linearized
time-dependent Vlasov equation for the non-adiabatic distribution
function perturbation $\tilde f(x,v)$ and integrating it:
$\tilde n =\int \tilde f dv$. The solution can be written[] in terms of
an integral over past time along the unperturbed orbit which is
\begin{equation}
  \label{eq:phim}
  \Phi_m(z,v,t)\equiv 
  \int_{-\infty}^t \hat\phi(z(\tau)){\rm e}^{-i\omega_m(\tau-t)}d\tau,
\end{equation}
where $\phi_1(y,z,t)=\hat\phi(z){\rm e}^{i(k_\perp y-\omega t)}$ is
the potential perturbation ($\ket{\hat\phi(z)}$ giving its parallel
dependence), its transverse wavenumber is $k_\perp$, and $z(\tau)$ is
the parallel position of the past unperturbed orbit that has velocity
$v$ at $(z,t)$. When a uniform magnetic field in the $z$-direction is
present with cyclotron frequency $\Omega$ and the background
perpendicular velocity distribution is Maxwellian, then the
distribution function perturbation is the following sum over cyclotron
harmonics[]
\begin{equation}\label{eq:ftmagnetic}
  \begin{split}
    \tilde f(z,y,v,t) &=  {\rm e}^{i(k_\perp y-\omega t)}
 \sum_{m=-\infty}^\infty i\left[\omega_m
  {\partial f_{\parallel0}\over \partial W_\parallel}
  +m\Omega {f_{\parallel0}\over T_\perp}\right]
  q_e\Phi_m {\rm e}^{-\xi_t^2}I_m(\xi_t^2)\\
  &= {\rm e}^{i(k_\perp y-\omega t)} \sum_{m=-\infty}^\infty \tilde f_m(z,v),
  \end{split}
\end{equation}
where $\xi_t^2=k^2T_\perp/\Omega^2m_e$, $I_m$ is the modified Bessel
function, and $\omega_m=m\Omega+\omega$. Therefore the contribution
for each harmonic is determined by the corresponding $\Phi_m$, and we
can regard each as giving a perturbed density contribution
$\tilde n_m(z)=\int \tilde f_m dv$. Eliding the $t$ and $y$-dependence
as implicit, we have
\begin{equation}
  \label{eq:13}
\tilde  n_m=\tilde{V}_{m}\ket{\hat\phi}=\tilde{V}_{m}\sum_u a_u\ket{u}=\int \sum_u a_u
  \tilde f_{um} dv=\sum_u a_u \tilde n_{um},
\end{equation}
where $\tilde f_{um}$ denotes $\tilde f_m$ with
$\ket{\hat\phi}=\ket{u}$ substituted in eq. (\ref{eq:ftmagnetic}), and
similarly $\tilde n_{um}=\int \tilde f_{um}dv$.

Solving analytically for $\Phi_m$ and $n_m$ in the non-uniform
equilibrium of an electron hole seems too difficult. The approach
taken here is to perform the required integrations numerically.

%We shall later require  the order of magnitude of $\tilde V\ket{4}$
%outside the hole.  In Hutchinson 2018,
%section 4.1 it is shown to be $\sim \omega_m^2$.


\subsection{The continuum contribution}

In the region far outside the hole, $|z/4|\gg 1$ where
$T=\tanh(z/4)\to\pm1$ and $S=\sinh(z/4)\to0$ in eq.\ (\ref{eq:2}), the
normalized continuum modes (taking them to be antisymmetrically
outward propagating) are
\begin{equation}
  \label{eq:14}
 \ket{a_p}= \ket{p}=A\etothe{ip|z/4|}=\sigma_z{(-15p^4+225p^2-120) +ip (p^{4}-85p^{2}+274)\over
      [8\pi(p^2+1^2)(p^2+2^2)(p^2+3^2)(p^2+4^2)(p^2+5^2)]^{1/2}}\etothe{ip|z/4|},
\end{equation}
where $\sigma_z={\rm sign}(z)$. Thus they are purely sinusoidal
waves. Assuming that for large enough $|z|$ the influence of the local
hole potential becomes negligible, such waves should there be
identified with the normal modes of the uniform background plasma.
The applicable normal mode for the present electrostatic approximation
when $\omega<\Omega,\omega_{pe}$ (and $\omega/k\gg v_{te}$) is the high
frequency end of the whistler branch of the cold plasma dispersion
relation at high refractive index $kc/\omega$. This is the obliquely
propagating ``magnetized langmuir wave'', for which at high
$\Omega/\omega_{pe}$, 
$\omega\approx\omega_{pe}\cos\theta$ (where
$k_\perp/k_\parallel=\tan\theta$). More completely, in the cold
electrostatic limit ignoring ion response, the dispersion relation
(including the upper hybrid waves) is
\begin{equation}
  \label{eq:15}
  k_\parallel^2/k_\perp^2={\omega^2(\Omega^2+\omega_{pe}^2-\omega^2)
    \over(\Omega^2-\omega^2)(\omega_{pe}^2-\omega^2)}.
\end{equation}
(In our normalized units $\omega_{pe}=1$.)  Assuming that $\omega$ and
$k_\perp$ are prescribed, there is only one $|k_\parallel|$ that
satisfies this dispersion relation; and corresponding to it, the
continuum eigenmode of $V_a$ has $p=4k_\parallel$.

Intuitively, then, the continuum contribution is limited more or less
to the single mode
$p/4=k_\parallel$ given by eq.\ (\ref{eq:15}).
For such a mode, one can immediately calculate the value of
$\Phi_m(z)$ for \emph{inward} orbit velocity, (negative $v$) on the
positive-$z$ side, using $z'=z(\tau)$ and $\tau=-(z-z')/v$, as
\begin{equation}
  \label{eq:16}
  \Phi_m(z) = \int_{\infty}^zA{\rm e}^{i[pz'/4+\omega_m
    (z-z')/v]}dz'/v
  ={A{\rm e}^{ipz/4} \over i(pv/4-\omega_m)}={A{\rm e}^{ik_\parallel z} \over
    i(k_\parallel v-\omega_m)},
\end{equation}
where dropping the infinity limit is justified by positive imaginary
part of $\omega$.  This $\Phi_m$ applies for all $z$ down to where the
eigenmode begins to deviate significantly, because of non-zero hole
potential, from the external wave; obviously that $z$ is of order a
few times 4. The non-adiabatic perturbation to \emph{outward}
(positive) orbit velocity, by contrast, is strongly affected by the
rapid variation of potential across the hole at $z\sim 0$ whose effect
is carried into the external region. However, that local hole
perturbation develops a phase shift further into the positive z region
that is equal to $z\omega_m/v$; and so different $v$ contributions to
the density $\tilde n$ arising in $\int \tilde f(v) dv$ oscillate
increasingly rapidly with $v$, as $z$ increases, cancelling out
eachother's contribution to $\tilde n$. The density integral, omitting the
common factor $\etothe{ik_\parallel z}$ and ignoring the width of the
hole, looks like
$(2/\sqrt{\pi})\int_0^\infty \etothe{-i\omega_m
  z/v}\etothe{-v^2/2}dv$.  It has its first $z$-zero near
$\omega_m z=1$. Thus, the density influence of the localized hole
perturbation extends approximately a normalized distance
$z_l\sim 1/\omega_m$ (dimensionally
$z_l\sim \lambda_{De}\omega_{pe}/\omega_m$) and becomes increasingly
attenuated further away. At $z>z_l$, moreover, the outward going
contribution to $\tilde n$ is increasingly contributed by the spatial
range $[z-z_l,z]$, and merges into being given by the same integral as
\ref{eq:16}.  Thus distant $\tilde n$ perturbations for inward and
outward going velocities are given approximately by integrating $dv$
the same $\Phi_m$ expression for $z>z_l$, but with $v$ of opposite
sign.

The magnitude of the non-adiabatic response in the wave region can be
understood intuitively by recognizing that the criterion for ignoring
Landau (or cyclotron) damping, $\omega_m/k\gg v_{te}$, is equivalent
to saying that the explicit \emph{time} dependence at the particle
($\partial/\partial t$) of the wave is far faster than the parallel
space dependence ($v\partial/\partial z$) at the typical
velocity. Thus, as a particle passes through the wave it mostly
experiences an increase or decrease of its total energy because of the
\emph{time}-variation of the wave potential, with little variation of
its velocity or kinetic energy. That is equivalent to saying that the
sum of adiabatic and non-adiabatic distribution variation is
\emph{small}; i.e.\ that they cancel each other to lowest order in
$k_\parallel v/\omega_m$. This is made algebraically explicit by
Taylor expanding the right hand side of (\ref{eq:16}) in the small
quantity $k_\parallel v/\omega_m$ and substituting into eq.\
(\ref{eq:ftmagnetic}), arriving at
\begin{equation}
  \label{eq:17}
  \tilde f_m = i\left[\omega_m
    {\partial f_{\parallel0}\over \partial W_\parallel}
    +m\Omega {f_{\parallel0}\over T_\perp}\right]
  q_e{\rm e}^{-\xi_t^2}I_m(\xi_t^2) A{\rm e}^{ik_\parallel z}
  \left[1+{k_\parallel v\over\omega_m}+
    \left({k_\parallel v\over\omega_m}\right)^2+\dots\right]
  \Big/(-i\omega_m).
\end{equation}
Setting aside the complexities of the cyclotron harmonic expansion for
the moment, if $m=0$ is all we need, the first (constant) term in the
expansion exactly cancels the adiabatic term
$q_eA{\partial f_{\parallel0}/\partial W_\parallel}$. The next term
(first-order) integrates to zero over the parallel velocity
distribution because of its velocity parity. The density change
associated with the magnetized Langmuir wave arises from the
\emph{second order} term in the Taylor expansion. The total
non-adiabatic Vlasov term $\tilde{V}\ket{p}$ has the same magnitude as
the adiabatic term $V_a\ket{p}$ outside the hole, and cannot be
ordered as ignorable there.

Instead though, the non-adiabatic density in the wave-like external region
becomes the potential there ($\ket{p}$) multiplied by a function of
($\omega_m$ and) $p$ that is \emph{independent of position}. Let us
express that in terms of an operator $\tilde{V}_{w}$. If only $m=0$
($\omega_m=\omega$) is
required and higher harmonic terms are negligible, then by inspection
of eq.\ (\ref{eq:17}), the operator
would be 
\begin{equation}
  \label{eq:20}
\tilde{V}_{w}\sim-q_e{dn_0\over d\phi_0}\left[1+\left(k_\parallel v_t\over\omega\right)^2\right]=1+\left(k_\parallel\over\omega\right)^2.
\end{equation}
However, that approximation effectively implements the limit
$\Omega\gg \omega_{pe}$, and gives a dispersion relation
$k_\parallel^2/k_\perp^2 = \omega^2/(\omega_{pe}^2-\omega^2)$, rather
than the full cold plasma expression for finite $\Omega/\omega$: eq.\
(\ref{eq:15}). The full expression arises when the $|m|=1$ terms in
the harmonic sum for the kinetic electrostatic dispersion
relation\footnote{See, e.g., Swanson (1989) section 4.4.} are also
included to lowest order in $\xi_t^2$. The resulting analytic form is
\begin{equation}
  \label{eq:24}
  \tilde V_w= 1 + \left(k_\parallel\over\omega\right)^2 +
  k_\perp^2 {T_\perp/T_\parallel\over\omega^2-\Omega^2}.
\end{equation}
The crucial point is that $(\tilde{V}-\tilde{V}_{w})\ket{q}$ is
localized, tending to zero in the region beyond $z_l$.

\subsection{Reducing the integral over the continuum}
\label{reducing}
Now we discuss mathematically how the concepts just introduced
intuitively are implemented.  Using $s\to q$ to observe our notation
for continuum modes, and supposing them to be normalized
($\bra{q}\ket{p}=\delta_{qp}$), we write eq.\ \ref{eq:10} explicitly
as an integral $a_pdp$ over an amplitude distribution function $a_p$
rather than a sum:
\begin{equation}
  \label{continw}
  0=(\lambda_q-\lambda_\perp)a_q+\bra{q}|\tilde{V}\ket{4}a_4
  +\int\bra{q}|\tilde{V}\ket{p}a_pdp.
\end{equation}
The adiabatic term is the first, which for normalized modes is
\begin{equation}
  \label{eq:contadiab}
  \int\bra{q}|V_a-\lambda_\perp\ket{p}a_pdp = \int
  (\lambda_p-\lambda_\perp)\delta_{qp}a_pdp = (\lambda_q-\lambda_\perp)a_q.
\end{equation}
The non-adiabatic contribution is $\bra{q}|\tilde{V}\ket{4}a_4$ plus the
continuum integral
\begin{equation}
  \label{eq:18}
  \int \bra{q}|\tilde{V}\ket{p}a_pdp=\int\bra{q}|\tilde{V}_{w}+(\tilde{V}-\tilde{V}_{w})\ket{p}a_pdp
  =\tilde{V}_{w}a_q +\int\bra{q}|(\tilde{V}-\tilde{V}_{w})\ket{p}a_pdp.
\end{equation}
As can be seen in Fig. \ref{modeplots}, the form of the continuum
modes in the inner region is almost independent of $p$. Actually for
small $p$, which is our interest here, the differences are even
smaller than that figure shows. Therefore, to that degree of
approximation, we can replace $\ket{p}$ with $\ket{q}$ in the final
term giving $\int\bra{q}|(\tilde{V}-\tilde{V}_{w})\ket{p}a_pdp\simeq
\bra{q}|\tilde{V}-\tilde{V}_{w}\ket{q}\int a_pdp$, and yielding the
equation
\begin{equation}
  \label{eq:21}
  (\lambda_\perp-\lambda_q-\tilde V_w)a_q= \bra{q}|\tilde{V}\ket{4}a_4
  +\bra{q}|\tilde{V}-\tilde{V}_{w}\ket{q}\int a_pdp.
\end{equation}
If $\int a_pdp$ is sufficiently small compared with $a_4$, the final
term can be ignored.  If, however, the term cannot be dropped, it has
weak dependence on $q$ locally, depending only on the integral
$\int a_pdp$. Let us for brevity denote it in eq.\ (\ref{eq:21}) as
$G$. The value of $a_q$ by contrast has resonant dependence on $q$
(i.e.\ on $k_\parallel$),
giving
\begin{equation}
  \label{eq:resonance}
  a_q = {\bra{q}|\tilde{V}\ket{4}a_4+G\over (\lambda_\perp-\lambda_q-\tilde{V}_{w})}
  ={\bra{q}|\tilde{V}\ket{4}a_4+G\over
    (k_\perp^2+k_\parallel^2+1-\tilde V_w) },
%  ={\bra{q}|\tilde{V}\ket{4}a_4+G\over k_\perp^2-(q/4)^2(1/\omega^2-1) },
\end{equation}
using eq.\ (\ref{eq:3}) $\lambda_q=-q^2/16-1=-k_\parallel^2-1$ and
$\lambda_\perp=k_\perp^2$.
%, and $\tilde{V}_{w}=[1+(k_\parallel/\omega)^2]=1+q^2/(16\omega^2)$
Let us write the denominator using eq.\ (\ref{eq:24})as
\begin{equation}
  \label{eq:25}
  k_\perp^2+k_\parallel^2+1-\tilde
  V_w\simeq k_\perp^2\left(1-{T_\perp/T_\parallel\over\omega^2-\Omega^2}\right)
  +k_\parallel^2\left(1-{1\over\omega^2}\right)
= {1\over4^2}\left({1\over\omega^2}-1\right)\left(q_0^2-q^2\right)
\end{equation}
where
\begin{equation}
  \label{eq:26}
  q_0=4k_\perp\sqrt{1+{T_\perp/T_\parallel\over\Omega^2-\omega^2}}
  \Bigg/\sqrt{{1\over\omega^2}-1}.
\end{equation}
The denominator is zero when $q^2=q_0^2$ which is the approximate
dispersion relation of the wave in a uniform plasma. It is equal to the cold
plasma expression (\ref{eq:15}) when $T_\perp/T_\parallel=1$.

In view of the resonant form of the expression for $a_q$, we can regard the
integral over this resonance $\int a_q dq$ as quantifying the total
continuum perturbation.  The integral $dq$ of the expression for $a_q$
just found (\ref{eq:resonance}) encounters the two poles of $a_q$ at
$q=\pm q_0$.
% =\pm4k_\perp\omega/\sqrt{1-\omega^2}$.
Near them the real part of the denominator becomes small. Its
imaginary part comes from the small, \emph{necessarily positive},
imaginary part of
$\omega=\omega_r+i\omega_i$. It can then quickly be shown that
$q_0$ has a small positive imaginary part: $q_0=q_{0r}+i\epsilon$.
\iftrue%
The infinite integral along the real
$q$-axis can be closed by returning in the upper part of the complex
plane ($\Im(q)$ positive) where the eigenmode tends
to zero. Its value is then just $2\pi i$ times the residue at the positive $q_0$ pole.
\begin{equation}
  \label{poles}
  \int a_q dq 
  ={2\pi i(\bra{q_0}|\tilde{V}\ket{4}a_4+G)
    \over
    -2q_{0}(1/\omega^2-1)/4^2}.
\end{equation}
\else
A resonant integral for small positive
$\epsilon$ and real $q$ generally gives
\begin{equation}
  \label{resint}
  \int_{-}^{+} {g(q) dq\over \pm q_{0r} \pm i\epsilon-q}\simeq \mp
  i\pi g(\pm q_{0r})
\end{equation}
for any slowly varying function $g$.
On this basis
\footnote{Write the integral $\int{g(q)dq\over
  (q_{0r}+i\epsilon)^2-q^2}=\int{g(q)dq\over 
(q_{0r}+i\epsilon-q)(q_{0r}+i\epsilon+q)}\simeq \int{g(q_{0r})dq\over
  (q_{0r}+i\epsilon-q)2q_{0r}}-\int{g(-q_{0r})dq\over
  2q_{0r}(-q_{0r}-i\epsilon-q) }\simeq -i\pi [g(q_{0r})+g(-q_{0r})]/2q_{0r}. $ when
 $\epsilon$ is
positive. This treatment is equivalent to closing the $q$-integration path
along $+i\infty$ and setting the closed contour integral equal to
$2\pi i$ times only the residue at $q=q_{0r}+i\epsilon$.}
\begin{equation}
  \label{poles}
  \int a_q dq \simeq -i4\pi {(\bra{q_{0r}}|\tilde{V}\ket{4}
    +\bra{-q_{0r}}|\tilde{V}\ket{4})a_4+2G
    \over
    2q_{0r}(1/\omega^2-1)/4^2
  }
  =\sum_{q=\pm q_{0r}}{-i4\pi(\bra{q}|\tilde{V}\ket{4}a_4+G)
    \over
    2q_{0r}(1/\omega^2-1)/4^2}.
\end{equation}
\fi
Taking the $G$ term to the left hand side and reinserting its definition, we get
\begin{equation}
  \label{eq:22}
  \left(1+{i2\pi \bra{q_{0}}|\tilde{V}-\tilde{V}_{w}\ket{q_{0}}\over
      2q_{0}(1/\omega^2-1)/4^2}\right)\int a_pdp
  ={-i2\pi\bra{q_0}|\tilde{V}\ket{4}a_4
    \over
    2q_{0}(1/\omega^2-1)/4^2}.
\end{equation}
and, for brevity, the approximately constant coefficient on the left will
be written
\begin{equation}
  \label{eq:23}
  C \equiv
  \left(1+{i2\pi\bra{q_{0}}|\tilde{V}-\tilde{V}_{w}\ket{q_{0}}\over
      2q_{0}(1/\omega^2-1)/4^2}\right).
\end{equation}

This analysis thus shows that the entire influence of the continuum
can (for given $\omega$ and $k_\perp$) be approximated by a single
parallel wavenumber $k_\parallel= q_{0}/4$ continuum mode, having an
amplitude $\int a_p dp$ given by eq.\ (\ref{eq:22}).

The continuum contribution to the eigenvalue in eq.\ (\ref{eq:12}) then
becomes
\begin{equation}
  \label{eigcon}
  {-i4\pi\bra{4}|\tilde V \ket{q_{0}}\bra{q_{0}}|\tilde{V}\ket{4}
    \over
    (q_{0}/4)(1/\omega^2-1)\bra{4}\ket{4}C}\iffalse
  =
  {-i4\pi\bra{4}|\tilde V \ket{q_{0}}\bra{q_{0}}|\tilde{V}\ket{4}
    \over
    k_\parallel(1/\omega^2-1)\bra{4}\ket{4}C}\fi.
\end{equation}
And the total eigenvalue equation, dropping symmetric discrete modes, becomes
\begin{equation}
  \label{eigtotal}
  \lambda_\perp=\lambda_4+{\bra{4}|\tilde{V}\ket{4}\over
    \bra{4}\ket{4}}
    +{\bra{4}|\tilde{V}\ket{2}\bra{2}|\tilde{V}\ket{4}\over
      (\lambda_\perp-\lambda_2)\bra{4}\ket{4} \bra{2}\ket{2}}
    +%\sum_{q=\pm q_{0}}
    {-i4\pi\bra{4}|\tilde V \ket{q_{0}}\bra{q_{0}}|\tilde{V}\ket{4}
    \over
    (q_{0}/4)(1/\omega^2-1)\bra{4}\ket{4}C}.
\end{equation}
When $\ket{4}$ and $\ket{2}$ are normalized as well as the
continuum modes, then $\bra{4}\ket{4}=\bra{2}\ket{2}=1$, of
course. Note that $q_0$ depends on $k_\perp$ which is
$\sqrt{\lambda_\perp}$, so both the last two terms depend on
$\lambda_\perp$.

Our reduction of the resonant continuum mode uses just the $m=0,\pm 1$
cyclotron harmonics to determine $q_{0}.$ That is justified by
supposing that the finite Larmor radius parameter
$\xi_t=k_\perp\sqrt{T_\perp/m_e}/\Omega$ is small, and recognizing
$I_m(\xi_t^2)\propto \xi_t^{2m}$. If $\xi_t$ is not small (because
$\Omega\not\gg k_\perp v_{t\perp}$), instead, the dependence
$\etothe{-\xi_t^2}$ predominates in eq.\ (\ref{eq:ftmagnetic}), and
harmonics up to $|m|\sim 3\xi_t$ approximate a continuous integral
over the perpendicular Maxwellian, with the relevant range of
$\omega_m$ approximately $\omega\pm 3k_\perp v_{t\perp}$. The
resultant frequency spread exceeds $\sim\omega$ for the whistler mode
(for which $\omega < \Omega$) implying that high harmonic
(perpendicular Landau) damping is strong. The continuum mode
contribution itself is then suppressed, and it is reasonable to ignore
coupling to the continuum modes.

\subsection{Numerical evaluation Probably Obsolete}

The terms in the dispersion relation (\ref{eigtotal}) contain inner
products of the form $\bra{s}|\tilde V\ket{u}$ that must be evaluated
by numerical integration. Those involving only $\ket{4}$ or $\ket{2}$
are performed using the same approach as has been described elsewhere
[]. They require spatial integrals over only the limited hole region where
those modes are non-negligible. However, the continuum mode remains
finite to large $z$; so $\bra{q_{0}}|\tilde{V}\ket{4}$ (and
$\bra{q_{0}}|\tilde{V}\ket{q_0}$ needed for the coefficient $C$)
require a much wider integration range, out to where the density
perturbation generated within the hole becomes negligible. These issues are
illustrated in Fig \ref{externfig}.
\begin{figure}
  \includegraphics[width=0.8\hsize]{extern1} (a)\\
  \includegraphics[width=0.8\hsize]{extern2} (b)
  \caption{Illustrating the process of calculating the inner products
    involving $\ket{4}$ and $\ket{q_0}$. Passing density contributions
    from positive velocity orbits (a) partially integrated to a low
    energy, (b) integrated over all relevant energies. The square
    point locates the edge of the inner hole region, chosen as
    $z=20$. The scaled $\Phi$ shown is for the last included energy
    $W$.\label{externfig}}
\end{figure}
Part way through the energy (positive parallel velocity) integration:
\ref{externfig}(a), the partial densities $\tilde n_q$ and
$\tilde n_4$ extend considerably into the external region, a distance
proportional to $v/\omega$. The eigenmode partial density $\tilde n_q$
differs mostly in its spatial oscillations from the wave partial
density $\tilde n_w$, which is the partial integral of
$\ket{q_0}/i(k_\parallel v -\omega)$. The $\tilde n_4$ density from
$\tilde V\ket{4}$ also has some oscillations arising from the
oscillation of its $\Phi$, of which the value at the maximum energy is
plotted (arbitrarily scaled). When the energy integration is complete:
\ref{externfig}(b), the oscillations have mostly been washed out by
integration, but there remain differences between $\tilde n_q$ and
$\tilde n_w$ close to the hole region that rapidly decrease further
away so that $\tilde n_q-\tilde n_w$ is limited in spatial
extent. Plot (b) includes the curve $V_w\ket{q_0}/2$, which represents
the simple multiplication of $\ket{q_0}$ by equation \ref{eq:24}. It
differs from $\tilde n_w$ because $\tilde n_w$ represents here only
outward velocity orbits, and so the first-order term of eq.\
\ref{eq:17} is not cancelled out in $\tilde n_w$. The result is not
exactly antisymmetric and does not exactly equal $V_w\ket{q_0}/2$ as
given by \ref{eq:24}. This difference illustrates physics, not an
error. A related feature is that $\tilde n_q$ and $\tilde n_w$ require
a non-zero value of the prior time integral $\Phi_m$ (e.q.\
\ref{eq:phim}) to be prescribed at the left-hand edge of the inner
region ($z=-z_j=-20$ in this illustration) representing the integral
along the past orbit from $z=-\infty$ to $-z_j$. This value is supplied
by eq.\ \ref{eq:16} for inward-propagating orbits:
$\ket{q_0}/i(-k_\parallel v -\omega)$ and is not exactly antisymmetric
with the outward moving values at $+z_j=20$. Denoting pedantically for
a moment the plotted densities, which use only positive velocities, as
$\tilde n_+$, the densities including the total velocity distribution
(positive and negative) are given by the antisymmetrizing sum
$\tilde n_{total}(z)=\tilde n_+(z)-\tilde n_+(-z)$. The external
values of $\tilde n_+$ at $|z|>z_j$ are not needed for inner products
with discrete modes such as $\bra{4}|\tilde V\ket{q_0}$ because they
($\bra{j}|$) are zero there. The antisymmetry of $\bra{j}|$ then means
we can simply double the inner products found using only
positive-velocity orbits to obtain the total.

Inner products with continuum modes $\bra{q_0}|$, however, require
integration for the external region $z_j<|z|$ (shown up to $\sim 450$ in
Fig.\ \ref{externfig}). Discrete mode external density, e.g.\
$\tilde V \ket{4}$, arises only from outgoing orbits, not incoming, so
the outgoing calculation for $\bra{q_0}|\tilde V\ket{4}$ need not be
doubled to account for both velocity signs, only to represent the
positive and negative $z$-ranges, since we integrate only over
$z>+z_j$. Incoming velocity orbits' density for continuum modes is not
negligible, but can accurately be taken as the wave value evaluated
with incoming sign (${\rm sign}(k_\parallel v)=-1$) in the factor
$(k_\parallel v-\omega)$.  \iffalse Thus, incoming density in the
outer region is equal to outgoing times the factor
$R=\int_0^\infty (k_\parallel v-\omega)/(-k_\parallel v-\omega)\;
f_\infty(v)dv$, and the total external contribution to
$\bra{q_0}|\tilde V\ket{q_0}$ is $1+R$ times the outgoing.  \fi For
present purposes, we really only need the difference
$\bra{q_0}|\tilde V-\tilde V_w\ket{q_0}$, and the two operators cancel
exactly for external incoming orbits, provided both are expressed in
terms of $(k_\parallel v-\omega)$ with incoming $v$-sign.  That is why
we calculate the outward going external density for both $\tilde V$
and $\tilde V_w$ using the outward $(k_\parallel v-\omega)$ factor,
rather than eq.\ (\ref{eq:24}). We double their
$\bra{q_0}|\tilde V\ket{q_0}$ values to account for positive and
negative $z$.



For finite $\Omega$, where the sum over harmonics is essential, we
determine $k_\parallel$ from eq.\ (\ref{eq:15}), the same for all
harmonics. Then for external passing orbits evaluating $\Phi_m$ as
$\ket{q_0}/i(k_\parallel v -\omega_m)$, integrating the resulting
$\tilde f_m$ $dv$ and summing over harmonics $m$ we get
$\tilde V_w\ket{q_0}$ externally. Such a treatment does not
apply to internal orbits, so instead we use for $\tilde V_w$ internally
eq.\ (\ref{eq:24}) multiplying $\ket{q_0}$. That excludes any resonances
between bounce frequency and higher-$m$ cyclotron harmonics. The other
$\tilde V\ket{4}$, $\tilde V\ket{2}$, $\tilde V\ket{q_0}$, calculations
simply substitute $\omega_m$ for $\omega$.

\subsection{External Analytic Integration}

In the external region $|z|>z_j$, the prior integral $\Phi$ giving
$\tilde f$ for modes $\ket{4}$ and $\ket{q_0}$ is non-zero for outward
moving orbits. We focus for definiteness on $z>z_j$ and positive $v$.
For inward moving orbits, by contrast, $\tilde f$ is zero externally for the
operations $\tilde V\ket{4}$ and $(\tilde V -\tilde
V_w)\ket{q_0}$. Consequently, the external contribution to the forces
$\bra{q_0}|\tilde V \ket{4}$ and $\bra{q_0}|\tilde V -\tilde V_m\ket{q_0}$
arises from the external integration $\int_{z_j}^\infty q_0^*(z) \Phi(z)
dz$ for outward orbits only.

In the external region, the velocity is constant; so for $\ket{4}$,
which has zero perturbed potential there, $\Phi$ is a simple time
delay factor times its value at the join between internal and external
regions $\Phi(z_j)$:
\begin{equation}
  \label{phi4}
  \Phi_{\ket{4}}(z)=\etothe{i\omega_m(z-z_j)/v}   \Phi_{\ket{4}}(z_j).
\end{equation}
The corresponding expression for the continuum mode, which has
non-zero external potential, may be
found by substituting the wave expression \ref{eq:14} for  $\ket{q_0}$
which can be integrated analytically to give an extra term, arriving
at 
\begin{equation}
  \label{phiq}
  \Phi_{\ket{q_0}}(z)=\etothe{i\omega_m(z-z_j)/v}   \Phi_{\ket{q_0}}(z_j)
  +{A\etothe{ik_\parallel z_j}\over i(k_\parallel v-\omega_m)}[\etothe{ik_\parallel(z-z_j)}
  -\etothe{i\omega_m(z-z_j)/v}].
\end{equation}
Recalling that the wave operator is simply constant, its external
contribution to $\Phi_{\ket{w}}$ for a particular positive velocity is
${A\etothe{ik_\parallel z}\over i(k_\parallel v-\omega_m)}$, which
exactly cancels the first term in the square bracket of
$\Phi_{\ket{q_0}}(z)$. Such cancellation is essential to produce a
finite value for $\bra{q_0}|\tilde V -\tilde V_m\ket{q_0}$. Hence, for
$(\tilde V -\tilde V_w)\ket{q_0}$,
\begin{equation}
  \label{phidiff}
  \Phi_{\ket{q_0-w}}(z)=\etothe{i\omega_m(z-z_j)/v}   \Phi_{\ket{q_0}}(z_j)
  %-\Phi_{w}(z_j)]
  -{A\etothe{ik_\parallel z_j}\over i(k_\parallel v-\omega_m)}
  \etothe{i\omega_m(z-z_j)/v}.
\end{equation}
To obtain the total external force we can carry out the inner product
$z$-integration analytically as
\begin{equation}
  \label{extforce}
  \int_{z_j}^\infty q_0^*(z) \Phi_u(z)  dz =
  {A^*\Phi_u(z_j)\etothe{-ik_\parallel z_j}\over i(k_\parallel  v-\omega_m)}
  \left[+ {|A|^2v\over (k_\parallel v-\omega_m)^2}\right],
\end{equation}
where $\Phi_u$ refers to the ``ket'' mode (4 or $q_0-w$ or for that
matter 2) and the $|A|^2$ term is present only if we are constructing
$\bra{q_0}|\tilde V -\tilde V_w)\ket{q_0}$. This force quantity is
multiplied by the extra weighting factors in equation
(\ref{eq:ftmagnetic}) and integrated (numerically) over parallel
velocity so as to produce the total inner product. Because the
integration is carried out only for positive $z$, the result must be
doubled to account for force exerted at negative $z$ (assuming
velocity symmetry of $f_0(v)$). The sum over relevant cyclotron
harmonics must also be carried out.

\section{Complete matrix formulation (Optional)}

Proceeding with no assumption about the size of cross coupling of secondary
modes, the eigenmode equation inner product with mode $\bra{s}|$ gives
\begin{equation}
  \label{eigengen}
  0=(\lambda_s-\lambda_\perp)\bra{s}\ket{s}a_s+\sum_j\bra{s}|\tilde{V}\ket{j}a_j
+\int\bra{s}|\tilde{V}\ket{p}a_pdp.
\end{equation}
We apply this equation for the three modes $s=4,2,q$. First for the
continuum mode ($\bra{s}|=\bra{q}|$) in the form,
\begin{equation}
  \label{eigq}
  (-\lambda_q+\lambda_\perp-V_w)a_q=\sum_j\bra{q}|\tilde{V}\ket{j}a_j
+\bra{q}|\tilde V-\tilde V_w\ket{q}\int a_pdp,
\end{equation}
which we integrate $dq$, recognizing the approximate independence of
the the right hand side of $q$ to find
\begin{equation}
  \label{eigqint}
  \int a_q dq =\int {dq\over \lambda_\perp-\lambda_q-V_w}
  \left(\sum_j\bra{q}|\tilde{V}\ket{j}a_j
+\bra{q}|\tilde V-\tilde V_w\ket{q}\int a_pdp \right).
\end{equation}
Thus, writing the resonant integral
$\int {dq\over \lambda_\perp-\lambda_q-V_w}=1/K$, where
$K\simeq-(1/\omega^2-1)q_0/(16\pi i) $, and substituting elsewhere the
value $q=q_0$ at the integrand's pole, we have
\begin{equation}
  \label{eigqint2}
  (K-\bra{q_0}|\tilde V-\tilde V_w\ket{q_0})\int a_pdp = \sum_j\bra{q_0}|\tilde{V}\ket{j}a_j.
\end{equation}
Using again the approximation that over the resonance
$\bra{j}|\tilde V\ket{p}$ can be regarded as independent of $p$, the
discrete modes then satisfy, for $l=4,2$,
\begin{equation}
  \label{eigenl}
  (\lambda_\perp-\lambda_l)a_l=\sum_j\bra{l}|\tilde{V}\ket{j}a_j
  +\int\bra{l}|\tilde{V}\ket{p}a_pdp
  =\sum_j\bra{l}|\tilde{V}\ket{j}a_j
  +\bra{l}|\tilde{V}\ket{q_0}\int a_pdp.
\end{equation}
Regarding the coefficients of the three modes 4,2,$q_0$ as composing a
column 3-vector whose transpose is
\begin{equation}
  \label{avector}
  \a^T=(a_4,a_2,\int a_pdp),
\end{equation}
the three relations in \ref{eigqint2} and \ref{eigenl} can be
considered a matrix equation $\M\a=0$, where
\begin{equation}
  \label{aveq}
  \M=\left[
  \begin{array}{ccc}
    \lambda_4-\lambda_\perp+\bra{4}|\tilde V\ket{4}
    &\bra{4}|\tilde V\ket{2}&\bra{4}|\tilde V\ket{q_0}\\
    \bra{2}|\tilde V\ket{4}
    &\lambda_2-\lambda_\perp+\bra{2}|\tilde
      V\ket{2}&\bra{2}|\tilde V\ket{q_0}\\
    \bra{q_0}|\tilde V\ket{4}
    &\bra{q_0}|\tilde V\ket{2}&-(K-\bra{q_0}|\tilde V-\tilde V_w\ket{q_0})
  \end{array}\right]
\iffalse
\left[
  \begin{array}{c}
    a_4\\
    a_2\\
    \int a_pdp
  \end{array}\right]=0.
\fi.
\end{equation}
The complex determinant $||\M||$ must be zero for a non-trivial
solution, and if $\omega$ is iterated until $||\M||=0$, the value
found will be the corresponding mode frequency, and the mode structure
will be given by the solution of $\M\a=0$. It is likely there are
multiple solutions, and one would like to ensure the correct one is
found. A possible approach might be to start at the $\omega$ that
satisfies the equation of a single $\ket{4}$ mode, namely that the top
left coefficient is zero:
$\lambda_4-\lambda_\perp+\bra{4}|\tilde V\ket{4}=0$.  The approach of
contouring the real and imaginary parts of the force
$\bra{4}|\tilde V\ket{4}-\lambda_\perp$ in the parameter ranges of
interest could also be adopted, except that for the full system one
should contour $||\M||$.

The ordered perturbative approximations found in prior sections correspond,
in this formulation, to neglecting $\bra{2}|\tilde V\ket{q_0}$,
$\bra{q_0}|\tilde V\ket{2}$ and $\bra{2}|\tilde V\ket{2}$; then
determining $a_2/a_4$ from the second row, and $\int a_pdp/a_4$ from the third
row.

\end{document}

%%% Local Variables:
%%% mode: latex
%%% TeX-master: t
%%% End:
