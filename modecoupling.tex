\documentclass[12pt]{article}

\usepackage[margin=1in]{geometry}
\usepackage{hyperref}
\usepackage{amsmath}
\usepackage{amssymb}
\usepackage{graphicx}

\def\ket#1{|#1\rangle}
\def\bra#1{\langle#1}
\def\sech{{\,\rm sech}}
\def\etothe#1{{\rm e}^{#1}}
%\def\Real{{\rm Re}}\def\Imag{{\rm Im}}

\begin{document}

\section{Introduction}

The stability of electron holes has recently been extensively studied
assuming that the most unstable perturbation is a parallel shift of
the hole position []. The eigenvalue, which determines the complex
frequency $\omega$, has previously been evaluated using a ``Rayleigh
Quotient'' which provides an approximation accurate to second order in
any deviations from the exact unstable mode structure. The results are
in reasonable agreement with PIC simulations, but some discrepancies
have been noted. The purpose of the present work is to discover
whether more accurate mode shape determination, including deviations
from the shift mode, can explain those discrepancies, and to quantify
by analysis how important the deviations are.

Two main phenomena have been observed in transverse instability
simulations that are not represented by the shift mode analysis. They
are (1) narrowing of the unstable mode structure relative to the shift
mode, when near marginal oscillatory stability; (2) generation of long
parallel length streaks or waves, often called ``whistlers'', that are
magnetized Langmuir waves during oscillatory instability at high
magnetic field.

A relatively compact approach to managing a generalization of the
unstable mode of a Vlasov-Poisson problem [] is to represent the
perturbation of potential in terms of the eigenmodes (which are
orthogonal) of a judiciously chosen differential operator. In the
present context the operator generally used represents the Poisson
equation for steady (or very slowly varying) potential
($\nabla^2\phi+n=0$ in appropriately normalized units). The particle
density $n_0(z)$ associated with the potential equilibrium
$\phi_0(z)$, can be determined from this equation. Then for
infinitessimally slow linearized potential perturbations $\phi_1$
about this equilibrium the perturbed density is
$n_1=\phi_1dn_0/d\phi_0$, and the resulting Poisson equation can be
written as the operator $V_a =(\nabla^2+dn_0/d\phi_0)$ acting on
$\phi_1$. This $V_a$ is called the ``adiabatic'' operator, and the
associated density perturbation $n_1$ the adiabatic density
(perturbation). The expansion of the potential perturbation is most
naturally in terms of the eigenmodes of $V_a$. For purely growing
instabilities $\omega=0+i\omega_i$, at the threshold $\omega_i\sim 0$,
evidently the adiabatic response is nearly equal to the total because
changes are infinitessimally slow; so the non-adiabatic part
$\tilde V$ of the operator is small compared with $V_a$ in the
perturbed Poisson equation. Consequently to lowest order the
perturbation unstable mode is equal to the eigenmode of $V_a$ with
zero eigenvalue. For a solitary potential structure in a uniform
background, such a zero eigenvalue always exists, and its eigenmode
has the form of a uniform shift of the equilibrium.

By the preceding argument, determining where $\omega_i=0$, i.e. the
stability threshold, can be accomplished exactly using just the shift
mode, provided that the real part $\omega_r$ of the mode frequency is
zero. However, some hole instabilities are oscillatory:
$\omega_r\not=0$, $\omega_i>0$. Then the marginal unstable mode is not
purely the shift mode, and the extent to which it includes
contributions from other eigenmodes of $V_a$ becomes an important
question. It can be explored by carrying the perturbation analysis to
first order in the other eigenmodes, in much the way that
time-independent perturbation theory is used in quantum
mechanics. This was successfully pursued in an early study of the
one-dimensional instability of a train of electron holes, which leads
eventually to hole merger[]. However, the few subsequent efforts to
apply it [] have been of limited utility either because of perturbing
about the wrong eigenmode, or, more fundamentally because of adopting
inappropriate approximations to the solution of the Vlasov equation,
which constitutes the complementary (and more difficult) part of the
Poisson-Vlasov system: the non-adiabatic perturbation.


\section{Eigenmode Expansion}
Let us adopt a minimalist bra-ket notation with the modes in
which we expand our solutions
denoted $\ket{u}$, where the range of $u$ being either real, $p,q,r,\dots$, or
integer, $j,l,m,\dots$, denotes respectively continuum or discrete modes. The
inner product denotes an overlap integral over (parallel) spatial
coordinate $\bra{u}\ket{s}\equiv \int u^*(x)s(x)dx$. Insofar as the
eigenmodes are orthonormal, we write $\bra{u}\ket{s}=\delta_{us}$,
where for continuum modes such that $u$ and $s$ are real parameters,
this is (approximately) a Dirac delta function,
$\delta_{us}=\delta(u-s)$, whereas for discrete modes $\delta_{jl}$ is
the Kronecker delta.

\subsection{Adiabatic Response Eigenmodes}
In scaled units Poisson's equation is $d^2\phi/dz^2=n-1$. It is
convenient for the eigenmode expansion to use an alternate parallel coordinate
$x=z/4$ so that the equilibrium is of the form
$\phi_0=\psi\sech^4(x)=\psi\sech^4(z/4)$.  The linearized equilibrium
(``adiabatic'') Poisson operator for a hole potential (writing for brevity
$S=\sech(x)$ and $T=\tanh(x)$ and denoting the $x$-derivative by
prime) can then be written
\begin{equation}
  \label{eq:1}
  V_a = \left[{d^2\over dx^2} - 4^2{dn\over d\phi}\right]
  = \left[{d^2\over dx^2} - {\phi^{'''}\over \phi'}\right]
  = \left[{d^2\over dx^2} +30S^2- 4^2\right].
\end{equation}
Notation $V$ reminds us that this is the part of the Vlasov operator,
transforming a potential into a density, and the subscript $a$ denotes
adiabatic (meaning steady). The factor $4^2$ multiplying the density arises
because of operating in $x$-space. [There might be a better choice of
notation simplification that avoided this double parallel notation;
but that needs more thought.]  The eigenmodes of this operator, satisfying
$V_a\ket{u}=\lambda \ket{u}$, can be written
\begin{equation}
  \label{eq:2}
  \begin{split}
 \ket{u}(x)= &\exp(\pm ux)\{[-15u^4 + (420S^2 - 225)u^2 - 945S^4 +
 840S^2 - 120]T \\
 &\qquad\qquad\pm u[u^4 + (-105S^2 + 85)u^2 + (945S^4 - 1155S^2
 +274)]\}
% \\ \equiv&  \exp(\pm ux)(P_uT\pm uQ_u)
.
  \end{split}
\end{equation}
Real $u$ values possess just five discrete values $j=1,2,3,4,5$ that
satisfy $\ket{j}(x)\to0$ as $x\to\infty$. In contrast, imaginary
$u$-values $u=ip$ give rise to the continuum modes $\ket{p}(x)$ which
are formally finite at infinity. The overlap integral $\bra{p}\ket{q}$
over a finite domain exists; and, as the domain tends to infinity, it tends
to a delta function.

The corresponding eigenvalues are
\begin{equation}
  \label{eq:3}
  \lambda_u= u^2-4^2=-p^2-4^2.
\end{equation}
The odd numbered discrete eigenmodes are symmetric (about $x=0$); and
the even numbered are antisymmetric. Normalized so that
$\bra{j}\ket{l}=\delta_{jl}$ they are given in Table \ref{discrete}. 
\begin{table}[ht]
  \center
\begin{tabular}{ccc}\
  Mode & Eigenvalue & Normalized Form\\
  $\ket{1}$& -15&$S(21S^4 - 28S^2 + 8)\sqrt{30}/16$\\
  $\ket{2}$& -12&$TS^2(3S^2 - 2)\sqrt{105}/4$\\
  $\ket{3}$& -7 &$-S^3(9S^2 - 8)\sqrt{105}/16 $\\
  $\ket{4}$&  0 &$-3TS^4\sqrt{70}/8 $\\
  $\ket{5}$&  9 &$3S^5\sqrt{35}/16 $\\
\end{tabular}
\caption{Discrete eigenmodes.\label{discrete}}
\end{table}
\noindent
In particular $\lambda_4=0$ for the shift mode
$\ket{4}\propto d\phi_0/dx$, and it is the the predominant
perturbation in essentially all linear instabilities. Moreover, it
couples only to antisymmetric modes; so $\ket{2}$ is the only other
discrete mode that needs to be considered.

The antisymmetric continuum modes are obtained by taking the sign to
reverse across $x=0$, i.e.\ $\pm\to \sigma_x={\rm signum}(x)$.  It can
be shown that continuum modes are ``normalized'' by dividing eq.\
(\ref{eq:2}) by the factor
\begin{equation}
  [2\pi(p^2+1^2)(p^2+2^2)(p^2+3^2)(p^2+4^2)(p^2+5^2)]^{1/2}, 
  \label{eq:4}
\end{equation}
in the sense that then $\bra{p}\ket{q}=\delta_{pq}=\delta(p-q)$. Far
from the hole ($x\gg1$) the normalized oscillatory continuum modes are
sinusoidal with amplitude $1/\sqrt{2\pi}=0.3989$. The eigenmodes are
plotted in Fig.\ \ref{modeplots}.
\begin{figure}\center
  \includegraphics[width=0.6\hsize]{continmode}
\caption{Eigenmodes of the adiabatic response operator.\label{modeplots}}
\end{figure}

\subsection{Including Non-adiabatic Response}
We now must consider the full linearized Poisson equation including the
non-adiabatic response arising from the solution to the time-dependent
Vlasov equation, $\tilde{V}$, as well as the adiabatic response
$V_a$. The form of the non-adiabatic Vlasov operator $\tilde{V}$ will
be discussed later. For a perturbation
$\phi_1(y,z,t)=\hat\phi(z){\rm e}^{i(k_\perp y-\omega t)}$ with
transverse wavenumber $k_\perp$ and frequency $\omega$, Poisson's
equation becomes
\begin{equation}
  \label{eq:5}
  (-16k_\perp^2+V_a+\tilde{V})\ket{\hat\phi}=0,
\end{equation}
in which it is convenient to regard $16k_\perp^2\equiv\lambda_\perp$ as the
full eigenvalue. 
We suppose its solution can be expanded as a sum of scalar
weights $w_u$ times the eigenmodes of $V_a$, $\ket{u}$:
\begin{equation}
  \label{eq:6}
  \ket{\hat\phi}=\sum_u w_u \ket{u},
\end{equation}
where the summation notation also includes an integral over the
continuum modes.

Then we can invoke the orthogonal properties of the
adiabatic eigenmodes and form the inner product
\begin{equation}
  \label{eq:7}
  \bra{s}|(%-16k_\perp^2+
  V_a+\tilde{V})\ket{\hat\phi}
  = %-16k_\perp^2+
  \lambda_s\bra{s}\ket{s}w_s+\sum_u\bra{s}|\tilde{V}\ket{u}w_u,
\end{equation}
which must be equal to $\lambda_\perp\bra{s}\ket{s}w_s$ to satisfy eq.\ \ref{eq:5}.
In particular, choosing the predominant mode $\bra{4}|$ for
$\bra{s}|$, for which $\lambda_4=0$, we get an eigenvalue equation
$0=(\lambda_4-\lambda_\perp)\bra{4}\ket{4}w_4+\sum_u\bra{4}|\tilde{V}\ket{u}w_u$.

If all the weights $w_u$ for $u\not=4$ are negligible, then
$\tilde{V}$ contributes a correction ($\lambda_{4}^{(1)}$) to the eigenvalue
\begin{equation}
  \label{eq:8}
  \lambda_{4}^{(1)}= {\bra{4}|\tilde{V}\ket{4}\over\bra{4}\ket{4}}=\lambda_\perp-\lambda_4
\end{equation}
The expression $\bra{4}|\tilde{V}\ket{4}/\bra{4}\ket{4}$ is the ``Rayleigh
quotient'' approximation for the eigenvalue of $V_a+\tilde{V}$
(because $\lambda_4=0$). It is
physically the (normalized) jetting force on particles because of the
unperturbed electric field $-d\phi_0/dx\propto\bra{4}|$, acting on the
non-adiabatic density perturbation $16 \tilde n = \tilde{V}\ket{4}$, integrated
over the entire hole. It balances the (normalized) Maxwell shear
stress from the transverse kinking of the hole ($16k_\perp^2$) to
make the total force zero.

Actually $\tilde{V}$ is a complicated nonlinear function of the complex
frequency $\omega$ of the mode; and for specified $k_\perp$ the
dispersion relation between $\omega$ and $k_\perp$ must be solved by
some kind of iterative procedure searching for an $\omega$ that
satisfies eq.\ (\ref{eq:8}). The imaginary part of $\omega$ thus found
determines the stability of the hole. This approximation has been
extensively pursued in the past and yielded stability results
that are in reasonable (but not perfect) agreement with simulation.
The question at hand is whether analysis can determine approximately
the magnitude of the other coefficients $w_u$ for $u\not=4$, and
therefore give a more accurate perturbation structure $\ket{\hat\phi}$
and $\omega$.

If instead of the approximation (\ref{eq:8}), we are able to evaluate all the
matrix coefficients $\bra{s}|\tilde{V}\ket{u}$ then in principle we can regard eq.\
(\ref{eq:7}) instead of (\ref{eq:8}), as a matrix eigen-system we must
solve to find the $\omega$ that permits a non-zero solution for the
vector $w_u$. The off-diagonal matrix entries $s\not=u$ are the
coupling of the potential modes by the non-adiabatic Vlasov operator
$\tilde{V}$.  The condition for the existence of a solution is that the
determinant of the matrix
$[(-\lambda_\perp+\lambda_s)\bra{s}\ket{s}\delta_{su}+\bra{s}|\tilde{V}\ket{u}]$ should
be zero. A possible iterative scheme might consist of finding its
eigenvalues, and adjusting $\omega$ until the eigenvalue for which
$w_4$ is predominant over the other coefficients becomes zero.

Such a program faces formidable practical challenges, however, because
each evaluation of $\tilde{V}\ket{u}$, requires a computation involving
multiple-dimension integrations over space and velocity distribution
--- repeated for each mode $\ket{u}$ and each adjustment of $\omega$.
Moreover, in principle, the continuum contains infinitely many modes,
and the matrix contains the square of the number of modes.  Obviously
we require this number to be reduced! Also, continuum modes $\ket{p}$
extend to $|z|=\infty$: an integration range that for computation
needs to be reduced.

Formally, any mode for which $\tilde{V}\ket{u}w_u$ is not small
must be retained in the sum of $\bra{s}|\tilde{V}\ket{u}$ in eq.\
\ref{eq:7}, giving for the dominant mode
\begin{equation}
  \label{eq:9}
  0=(\lambda_4-\lambda_\perp)\bra{4}\ket{4}w_4+\bra{4}|\tilde{V}\ket{4}w_4+\sum_{u\not=4}\bra{4}|\tilde{V}\ket{u}w_u,
\end{equation}
but also for $s\not=4$
\begin{equation}
  \label{eq:10}
 0=(\lambda_s-\lambda_\perp)\bra{s}\ket{s}w_s+\bra{s}|\tilde{V}\ket{4}w_4+\sum_{u\not=4}\bra{s}|\tilde{V}\ket{u}w_s.
\end{equation}


The well known approach of time-independent perturbation theory in
elementary quantum mechanics (e.g. Dirac Section 43) regards $\tilde{V}$ as
systematically small, and takes $w_s$ for $s\not=4$ also to be first
order small relative to $w_4$. The first order approximation of eq.\
\ref{eq:10} then drops the final sum term, giving
\begin{equation}
  \label{eq:11}
  w_s = {\bra{s}|\tilde{V}\ket{4}w_4\over (\lambda_\perp-\lambda_s)\bra{s}\ket{s}}.
\end{equation}
Substituting back (with $s\to u$) into eq.\ \ref{eq:9}, the
eigenvalue to second order is
\begin{equation}
  \label{eq:12}
 \lambda_\perp\simeq \lambda_4+\lambda_{4}^{(1)}+\lambda_{4}^{(2)}=\lambda_4
  +{\bra{4}|\tilde{V}\ket{4}\over \bra{4}\ket{4}}
    +\sum_{u\not=4}
    {\bra{4}|\tilde{V}\ket{u}\bra{u}|\tilde{V}\ket{4}\over
      (\lambda_\perp-\lambda_u)\bra{4}\ket{4}\bra{u}\ket{u}}.
\end{equation}
Normally the substitution $\lambda_\perp\simeq\lambda_4$ is made in
the final sum; but we do not need to do that since we consider
$\lambda_\perp$ to be given and $\omega$ to be changed to achieve
equality in this equation.

Equations (\ref{eq:11}) and (\ref{eq:12}) apply fully for coupled
discrete modes $\ket{j}$, giving the first order mode amplitude and
second order eigenvalue correction from them.  However, in our case,
for the continuum modes, $\tilde{V}$ is not systematically small
everywhere, and an altered approach appropriate to the actual form of
$\tilde{V}$ must be adopted.


\section{Calculating the Eigenmode coefficients}
\subsection{The Non-adiabatic Linearized Vlasov Operator}

The operator $\tilde{V}$ transforms a potential perturbation $\ket{u}$ into
a non-adiabatic density perturbation $\tilde n$, both of which are
functions of $x$. It does so by solving the time-dependent Vlasov
equation for the non-adiabatic distribution function perturbation
$\tilde f(x,v)$ and integrating it: $\tilde n =\int \tilde f dv$. The
linearized solution can be written in terms of an integral over past
time along the unperturbed orbit which is
\begin{equation}
  \label{eq:phim}
  \Phi_m(z,v,t)\equiv 
  \int_{-\infty}^t \hat\phi(z(\tau)){\rm e}^{-i\omega_m(\tau-t)}d\tau,
\end{equation}
where $\phi_1(y,z,t)=\hat\phi(z){\rm e}^{i(k_\perp y-\omega t)}$ is the potential
perturbation, and $z(\tau)$ is the past unperturbed orbit having
velocity $v$ at $(z,t)$. When a uniform magnetic field in the
$z$-direction is present with cyclotron frequency $\Omega$, the
perturbation has transverse wavenumber $k_\perp$, and the background
perpendicular velocity distribution is Maxwellian, then the
distribution function perturbation is[] the following sum over cyclotron
harmonics
\begin{equation}\label{eq:ftmagnetic}
  \begin{split}
    \tilde f(z,y,v,t) &=  {\rm e}^{i(k_\perp y-\omega t)}
 \sum_{m=-\infty}^\infty i\left[\omega_m
  {\partial f_{\parallel0}\over \partial W_\parallel}
  +m\Omega {f_{\parallel0}\over T_\perp}\right]
  q_e\Phi_m {\rm e}^{-\xi_t^2}I_m(\xi_t^2)\\
  &= {\rm e}^{i(k_\perp y-\omega t)} \sum_{m=-\infty}^\infty \tilde f_m(z,v),
  \end{split}
\end{equation}
where $\xi_t^2=k^2T_\perp/\Omega^2m_e$, $I_m$ is the modified Bessel
function, and $\omega_m=m\Omega+\omega$. Therefore the contribution
for each harmonic is determined by the corresponding $\Phi_m$, and we
can regard each as giving a perturbed density contribution
$\tilde n_m(z)=\int \tilde f_m dv$. Eliding the $t$ and $y$-dependence
as implicit, we have
\begin{equation}
  \label{eq:13}
\tilde  n_m={\scriptstyle {1\over16}}\tilde{V}_{m}\ket{\hat\phi}={\scriptstyle{1\over16}}\tilde{V}_{m}\sum_u w_u\ket{u}=\int \sum_u w_u
  \tilde f_{um} dv=\sum_u w_u \tilde n_{um},
\end{equation}
where $\tilde f_{um}$ denotes $\tilde f_m$ with
$\ket{\hat\phi}=\ket{u}$ substituted in eq. (\ref{eq:ftmagnetic}), and
similarly $\tilde n_{um}=\int \tilde f_{um}dv$.

We shall later require  the order of magnitude of $\tilde V\ket{4}$
outside the hole.  In Hutchinson 2018,
section 4.1 it is shown to be $\sim 16\omega_m^2$.


\subsection{The continuum contribution}

In the region far outside the hole, $|x|\gg 1$ where
$T=\tanh(x)\to\pm1$ and $S=\sinh(x)\to0$ in eq.\ (\ref{eq:2}), the
normalized continuum modes (taking them to be antisymmetrically
outward propagating) are
\begin{equation}
  \label{eq:14}
 \ket{w_p}= \ket{p}=A\etothe{ip|x|}=\sigma_x{(-15p^4+225p^2-120) +ip (p^{4}-85p^{2}+274)\over
      [2\pi(p^2+1^2)(p^2+2^2)(p^2+3^2)(p^2+4^2)(p^2+5^2)]^{1/2}}\etothe{ip|x|},
\end{equation}
where $\sigma_x=signum(x)$. Thus they are purely sinusoidal
waves. Assuming that for large enough $|x|$ the influence of the local
hole potential becomes negligible, such waves should there be
identified with the normal modes of the uniform background plasma.
The applicable normal mode for the present electrostatic approximation
when $\omega<\Omega,\omega_{pe}$ (and $\omega/k\gg v_{te}$) is the high
frequency end of the whistler branch of the cold plasma dispersion
relation at high refractive index $N=kc/\omega$. This is the obliquely
propagating ``magnetized langmuir wave'', for which
$\omega\approx\omega_{pe}\cos\theta$ (where
$k_\perp/k_\parallel=\tan\theta$) or equivalently
\begin{equation}
  \label{eq:15}
  k_\parallel^2/k_\perp^2=\omega^2/(\omega_{pe}^2-\omega^2).
\end{equation}
(In our normalized units $\omega_{pe}=1$.)  Assuming that $\omega$ and
$k_\perp$ are prescribed, there is only one $|k_\parallel|$ that
satisfies this dispersion relation; and corresponding to it, the
continuum eigenmode of $V_a$ has $p=4k_\parallel$ (using $x=z/4$).

Intuitively, then, the continuum contribution is limited more or less
to the single mode
$p/4=k_\parallel= [k_\perp^2\omega^2/(\omega_{pe}^2-\omega^2)]^{1/2}$.
For such a mode, one can immediately calculate the value of
$\Phi_m(z)$ for \emph{inward} orbit velocity, (negative $v$) on the
positive-$x$ side, using $z'=z(\tau)$ and $\tau=-(z-z')/v$, as
\begin{equation}
  \label{eq:16}
  \Phi_m(z) = \int_{\infty}^zA{\rm e}^{i[pz'/4+\omega_m
    (z-z')/v]}dz'/v
  ={A{\rm e}^{ipx} \over i(pv/4-\omega_m)}={A{\rm e}^{ik_\parallel z} \over
    i(k_\parallel v-\omega_m)},
\end{equation}
where dropping the infinity limit is justified by positive imaginary
part of $\omega$.  This $\Phi_m$ applies for all $z$ down to where the
eigenmode begins to deviate significantly, because of non-zero hole
potential, from the external wave; obviously that $z$ is of order 4
(i.e.\ $x\sim 1$). The non-adiabatic perturbation to \emph{outward}
(positive) orbit velocity, by contrast, is strongly affected by the
rapid variation of potential across the hole at $z\sim 0$ whose effect
is carried into the external region. However,
that local hole perturbation develops a phase shift further into the
positive z region that is equal to $z\omega_m/v$; and so different $v$
contributions to the density $\tilde n$ arising in
$\int \tilde f(v) dv$ oscillate increasingly rapidly with $v$, as $z$
increases, cancelling out eachother's contribution to $\tilde n$. This
integral, omitting the common factor $\etothe{ik_\parallel z}$ and
ignoring the width of the hole, looks like
$(2/\sqrt{\pi})\int_0^\infty \etothe{-i\omega_m
  z/v}\etothe{-v^2/2}dv$.  It has its first $z$-zero near
$\omega_m z=1$. Thus, the density influence of the localized hole
perturbation extends approximately a normalized distance
$z_l\simeq 1/\omega_m$ (dimensionally
$z_l\simeq \lambda_{De}\omega_{pe}/\omega_m$) and becomes increasingly
attenuated further away. At $z>z_l$, moreover, the outward going
contribution to $\tilde n$ is increasingly contributed by the spatial
range $[z-z_l,z]$, and merges into being given by the same integral as
\ref{eq:16}.  Thus distant $\tilde n$ perturbations for inward and outward
going velocities are given approximately by integrating $dv$ the
same $\Phi_m$ expression for $z>z_l$, but with $v$ of opposite sign.

The importance of the non-adiabatic response in the wave region can be
understood intuitively by recognizing that the criterion for ignoring
Landau (or cyclotron) damping, $\omega_m/k\gg v_{te}$, is equivalent
to saying that the explicit \emph{time} dependence at the particle
($\partial/\partial t$) of the wave is far faster than the parallel
space dependence ($v\partial/\partial z$) at the typical
velocity. Thus, as a particle passes through the wave it mostly
experiences an increase or decrease of its total energy because of the
\emph{time}-variation of the wave potential, with little variation of
its velocity or kinetic energy. That is equivalent to saying that the
sum of adiabatic and non-adiabatic distribution variation is
\emph{small}; i.e.\ that they cancel each other to lowest order in
$k_\parallel v/\omega_m$. This is made algebraically explicit by
Taylor expanding the right hand side of (\ref{eq:16}) in the small
quantity $k_\parallel v/\omega_m$ and substituting into eq.\
(\ref{eq:ftmagnetic}), arriving at
\begin{equation}
  \label{eq:17}
  \tilde f_m = i\left[\omega_m
    {\partial f_{\parallel0}\over \partial W_\parallel}
    +m\Omega {f_{\parallel0}\over T_\perp}\right]
  q_e{\rm e}^{-\xi_t^2}I_m(\xi_t^2) A{\rm e}^{ipx}
  \left[1+{k_\parallel v\over\omega_m}+
    \left({k_\parallel v\over\omega_m}\right)^2+\dots\right]
  \Big/(-i\omega_m).
\end{equation}
Setting aside the complexities of the cyclotron harmonic expansion
for the moment, if $m=0$ is all we need, the first (constant) term in
the expansion exactly cancels the adiabatic term
$q_eA{\partial f_{\parallel0}/\partial W_\parallel}$. The next
term (first-order) integrates to zero over the parallel velocity
distribution because of its velocity parity. The density change
associated with the magnetized Langmuir wave arises from the
\emph{second order} term in the Taylor expansion. The Vlasov term
$\tilde{V}\ket{p}$ is the same magnitude as the adiabatic term $V_a\ket{p}$
outside the hole, and cannot be ordered as ignorable there.

Instead though, the non-adiabatic density in the wave-like external region
becomes the potential there ($\ket{p}$) multiplied by a function of
($\omega_m$ and) $p$ that is \emph{independent of position}. Let us
express that in terms of an operator $\tilde{V}_{w}$ where
\begin{equation}
  \label{eq:20}
\tilde n_{w}={1\over16}\tilde{V}_{w}=-q_e{dn_0\over d\phi_0}\left[1+\left(k_\parallel v_t\over\omega_m\right)^2\right]=\left[1+\left(k_\parallel\over\omega_m\right)^2\right].  
\end{equation}
 Then
$\tilde{V}-\tilde{V}_{w}$ is a localized operator tending to zero in
the region beyond $z_l$. Now we demonstrate mathematically the concepts
just introduced intuitively.

\subsection{Reducing the integral over the continuum}
\label{reducing}
Using $s\to q$ to observe our notation for continuum modes, and
supposing them to be normalized ($\bra{q}\ket{p}=\delta_{qp}$), we write eq.\
\ref{eq:10} explicitly as an integral $w_pdp$ over a
weight distribution function $w_p$ rather than a
sum:
\begin{equation}
  \label{continw}
  0=(\lambda_q-\lambda_\perp)w_q+\bra{q}|\tilde{V}\ket{4}w_4
  +\int\bra{q}|\tilde{V}\ket{p}w_pdp.
\end{equation}
The adiabatic term is the first, which for normalized modes is
\begin{equation}
  \label{eq:contadiab}
  \int\bra{q}|V_a-\lambda_\perp\ket{p}w_pdp = \int
  (\lambda_p-\lambda_\perp)\delta_{qp}w_pdp = (\lambda_q-\lambda_\perp)w_q.
\end{equation}
The non-adiabatic contribution is $\bra{q}|\tilde{V}\ket{4}w_4$ plus the
continuum integral
\begin{equation}
  \label{eq:18}
  \int \bra{q}|\tilde{V}\ket{p}w_pdp=\int\bra{q}|\tilde{V}_{w}+(\tilde{V}-\tilde{V}_{w})\ket{p}w_pdp
  =\tilde{V}_{w}w_q +\int\bra{q}|(\tilde{V}-\tilde{V}_{w})\ket{p}w_pdp.
\end{equation}
As can be seen in Fig. \ref{modeplots}, the form of the continuum
modes in the inner region is almost independent of $p$. Actually for
small $p$, which is our interest here, the differences are even
smaller than that figure shows. Therefore, to that degree of
approximation, we can replace $\ket{p}$ with $\ket{q}$ in the final
term giving $\int\bra{q}|(\tilde{V}-\tilde{V}_{w})\ket{p}w_pdp\simeq
\bra{q}|\tilde{V}-\tilde{V}_{w}\ket{q}\int w_pdp$, and yielding the
equation
\begin{equation}
  \label{eq:21}
  (\lambda_q-\lambda_\perp-\tilde V_w)w_q= \bra{q}|\tilde{V}\ket{4}w_4
  +\bra{q}|\tilde{V}-\tilde{V}_{w}\ket{q}\int w_pdp.
\end{equation}
If $\int w_pdp$ is sufficiently small compared with $w_4$, the final
term can be ignored.  If, however, the term cannot be dropped, it has
weak dependence on $q$ locally, depending only on the integral
$\int w_pdp$. Let us for brevity denote it in eq.\ (\ref{eq:21}) as
$G$. The value of $w_4$ by contrast has resonant dependence on $q$,
giving
\begin{equation}
  \label{eq:resonance}
  w_q = {\bra{q}|\tilde{V}\ket{4}w_4+G\over (\lambda_\perp-\lambda_q-\tilde{V}_{w})}
  ={\bra{q}|\tilde{V}\ket{4}w_4+G\over
    16(k_\perp^2+k_\parallel^2-k_\parallel^2/\omega^2) }
  ={\bra{q}|\tilde{V}\ket{4}w_4+G\over 16k_\perp^2-q^2(1/\omega^2-1) },
\end{equation}
using (eq.\ \ref{eq:3}) $\lambda_q=-q^2-4^2$ , $\lambda_\perp=16k_\perp^2$, and
$\tilde{V}_{w}=16[1+(k_\parallel/\omega)^2]=16+q^2/\omega^2$. Notice that the
denominator is zero when $\omega=k_\parallel/k=\cos\theta$, the
dispersion relation of the wave in a uniform plasma. 

In view of the resonant form of this expression, we can regard the
integral over this resonance $\int w_q dq$ as quantifying total
continuum perturbation.  The integral $dq$ of the expression for $w_q$
just found (\ref{eq:resonance}) encounters the two poles of $w_q$ at
$q=\pm q_0=\pm4k_\perp\omega/\sqrt{1-\omega^2}$. Near them the real
part of the denominator becomes small. Its imaginary part comes from
the small, \emph{necessarily positive}, imaginary part of
$\omega=\omega_r+i\omega_i$. Then
$16k_\perp^2-q^2(1/\omega^2-1)\simeq 16k_\perp^2-q^2(1/\omega_r^2-1)
+iq^2/\omega_r^22(\omega_i/\omega_r)$.  A resonant integral for small
positive $\epsilon$ generally gives
\begin{equation}
  \label{resint}
  \int_{-}^{+} {g(q) dq\over q_0\pm i\epsilon-q}\simeq \mp i\pi g(q_0)
\end{equation}
for any slowly varying function $g$. On this basis
\iftrue
\footnote{Write the integral $\int{g(q)dq\over
  (q_0+i\epsilon)^2-q^2}=\int{g(q)dq\over 
(q_0+i\epsilon-q)(q_0+i\epsilon+q)}\simeq \int{g(q_0)dq\over
  (q_0+i\epsilon-q)2q_0}-\int{g(-q_0)dq\over
  2q_0(-q_0-i\epsilon-q) }\simeq -i\pi [g(q_0)+g(-q_0)]/2q_0. $ when
 $\epsilon$ is
positive. This treatment is equivalent to closing the $q$-integration path
along $+i\infty$ and setting the closed contour integral equal to
$2\pi i$ times only the residue at $q=q_0+i\epsilon$.}
\fi
,
\begin{equation}
  \label{poles}
  \int w_q dq \simeq -i\pi {(\bra{q_0}|\tilde{V}\ket{4}
    +\bra{-q_0}|\tilde{V}\ket{4})w_4+2G
    \over
    2q_0(1/\omega_r^2-1)
  }
  =\sum_{q=\pm q_0}{-i\pi(\bra{q}|\tilde{V}\ket{4}w_4+G)
    \over
    2q_0(1/\omega_r^2-1)}.
\end{equation}
[Should we write $1/\omega^2$ instead of $1/\omega_r^2$?]
Taking the $G$ term to the left hand side and reinserting its definition, we get
\begin{equation}
  \label{eq:22}
\left(1+{i\pi \bra{q_0}|\tilde{V}-\tilde{V}_{w}\ket{q_0}\over q_0(1/\omega_r^2-1)}\right)\int w_pdp  =\sum_{q=\pm q_0}{-i\pi\bra{q}|\tilde{V}\ket{4}  \over
    2q_0(1/\omega_r^2-1)}w_4;
\end{equation}
and, for brevity, the approximately constant coefficient on the left will
be written
\begin{equation}
  \label{eq:23}
  C \equiv \left(1+{i\pi\bra{q_0}|\tilde{V}-\tilde{V}_{w}\ket{q_0}\over q_0(1/\omega_r^2-1)}\right).
\end{equation}

The continuum contribution to the eigenvalue in eq.\ (\ref{eq:12}) then
becomes (approximately)
\begin{equation}
  \label{eigcon}
  \sum_{q=\pm q_0} {-i\pi\bra{4}|\tilde V \ket{q}\bra{q}|\tilde{V}\ket{4}
    \over
    2q_0(1/\omega_r^2-1)\bra{4}\ket{4}C}
  ={-i\pi\bra{4}|\tilde V \ket{q_0}\bra{q_0}|\tilde{V}\ket{4}
    \over
    q_0(1/\omega_r^2-1)\bra{4}\ket{4}C},
\end{equation}
using symmetry; and the total eigenvalue equation, dropping symmetric discrete modes, becomes
\begin{equation}
  \label{eigtotal}
  \lambda_\perp=\lambda_4+{\bra{4}|\tilde{V}\ket{4}\over
    \bra{4}\ket{4}}
    +{\bra{4}|\tilde{V}\ket{2}\bra{2}|\tilde{V}\ket{4}\over
      (\lambda_\perp-\lambda_2)\bra{4}\ket{4} \bra{2}\ket{2}}
    +%\sum_{q=\pm q_0}
    {-i\pi\bra{4}|\tilde V \ket{q_0}\bra{q_0}|\tilde{V}\ket{4}
    \over
    q_0(1/\omega_r^2-1)\bra{4}\ket{4}C}.
\end{equation}
When $\ket{4}$ and $\ket{2}$ are normalized as well as the
continuum modes, then $\bra{4}\ket{4}=\bra{2}\ket{2}=1$, of
course.

\subsection{Sum over cyclotron harmonics}

Section \ref{reducing} is written as if only one cyclotron harmonic
($m=0$) is needed. But it can in fact be applied separately to each
needed harmonic ($m\not=0$), interpreting each occurrence of $\omega$
as $\omega_m$. Each harmonic then has a different value of
$q_0=q_{0m}=4k_\perp\omega_m/\sqrt{1-\omega_m^2}$, with a different
resonant wave number of the external continuum. But each harmonic has
a weight still given by eq.\ (\ref{poles}) and a contribution
(\ref{eigcon}) to the total eigenvalue. Provided that no additional
resonances are introduced by the harmonic sum, the eigenmode and
eigenvalue contributions are simply summed.

When $\xi_t=k\sqrt{T_\perp/m_e}/\Omega$ is small, very few cyclotron
harmonics are required because $I_m(\xi_t^2)\propto \xi_t^{2m}$. When
$\xi_t$ is large, instead, the dependence $\etothe{-\xi_t^2}$
predominates in eq.\ (\ref{eq:ftmagnetic}), and many harmonics up to
$\sim 3\xi_t$ approximate a continuous integral over the perpendicular
Maxwellian. In either case, the relevant range of $\omega_m$ is
approximately $\omega\pm 3k_\perp v_{t\perp}$. Therefore, when
$T_\perp \sim 1$ (near isotropy of the external distribution giving
$ v_{t\perp}\sim 1$), and $k_\perp\ll 1$ (long perpendicular wavelength
compared to hole length) the main effect is to broaden the
$q$-resonance by a width
$\delta \omega\sim3k_\perp v_{t\perp}\ll 1 $. We will assume
that this effect, which is represented in our calculation, is
sufficient and that cyclotron resonances are unimportant.


\subsection{Evaluating $\bra{4}|\tilde{V}\ket{q}$ and $\bra{q}|\tilde{V}\ket{4}$}

All the needed inner product integrals over $z$ are limited to the
hole region where the discrete modes are nonzero, except for
$\bra{q}|\tilde V\ket{4}$.  The numerical scheme for the hole region
is identical to code previously used for a single mode ($\ket{4}$)
analysis [], but expanded to calculate the required additional terms
in eq.\ (\ref{eigtotal}), consisting of products between $\ket{4}$
and the auxiliary modes $\ket{2}$ and $\ket{q}$.  The exception,
$\bra{q}|\tilde V\ket{4}$, requires integration beyond the hole since
the outward moving orbits cause density perturbation there, even from
potentials localized to the hole. The the continuum potential mode
$\bra{q}|$ is not limited in extent; so numerical evaluation of
$\bra{q}|\tilde V\ket{4}$ requires $z$ integrals to $\gtrsim z_l$,
which is much greater than the hole extent.


The non-adiabatic density $\tilde V \ket{4}/16$ in the external region
arises only from orbits that are moving outward; that is: when the
sign of $v$ and $x$ are the same. Therefore the external contribution
to $\bra{q}|\tilde V \ket{4}$ can be derived on the positive side of
the hole from a positive-going velocity integration over orbits that
start at the positive hole edge, taking the $\Phi_m$ there as initial
condition, and then continuing the spatial integral for greater $z$.
Then by symmetry we can simply double this contribution to account for
the outgoing orbits of opposite $z$ (and $v$) sign, and add the
internal (trapped and passing) hole integral, to give the total
$\bra{q}|\tilde V \ket{4}$.  The external integration is calculated on
a much coarser mesh than the hole integrals, with comparable mesh
count, but extending sufficiently beyond $z_l$.

The inner products for prescribed $\omega$ and $k$ give the eigenvalue
from eq.\ (\ref{eigtotal}) and the total eigenmode as the sum eq.\
\ref{eq:6} with $w_2/w_4$ and $w_q/w_4$ given by \ref{eq:11} and
\ref{eq:resonance} respectively. 



\appendix
\section{Ignorability of $\bra{q}|(\tilde{V}-\tilde{V}_{w})\ket{q}$}
\label{ignorability}

In figure \ref{ntildeplots} the processes of reduction of the
eigenvalue equations are illustrated by presenting example perturbed
densities calculated numerically. The key requirement is to remove the
otherwise large $\bra{q}|\tilde V\ket{q}$ by subtracting from
$\tilde V$ a wave-mode represented by $\tilde V_w$.
\begin{figure}[htp]
\center  \includegraphics[width=0.7\hsize]{ntildeplots}
  \caption{Illustrative figure of contributions to eigenmode
    equations. \label{ntildeplots}}
\end{figure}
The challenge is that outside the hole ($z\gg 4$) the
continuum-generated non-adiabatic density perturbation
$\tilde n_q=\tilde V\ket{q}/16$ remains large; so an inner product
$\bra{q}|\tilde V\ket{q}$ (integrating $\int^\infty dz$) has large
contributions from the external region. It is not small. However, the
corresponding wave mode density arising from applying the
$z$-independent operator
$\tilde V_w=\int_0^\infty dv {i\over(k_\parallel v-\omega)}$ to
$\ket{q}$ and multiplying by the leading factors of eq.\ \ref{eq:17},
yielding $\tilde n_w$ is an excellent approximation to $\tilde n$ in
the external region, as the top panel shows. Consequently, even though
the wave mode $\bra{q}|$ is non-negligible to near infinity, its
inner product $\bra{q}|(\tilde{V}-\tilde{V}_{w})\ket{q}$ is finite,
and (as the numerics demonstrates) of approximately the same amplitude
as $\bra{q}|\tilde V\ket{4}$. The density corresponding to
$|\tilde V\ket{4}/16$ is shown in the middle panel, and the continuum
mode for reference in the bottom panel.

\iffalse
The continuum mode equation is

\begin{equation}
  \label{eq:19}
  0= (\lambda_q-\lambda_\perp)w_q+\bra{q}|\tilde{V}\ket{4}w_4
  +\tilde{V}_{w}w_q
  +\bra{q}|(\tilde{V}-\tilde{V}_{w})\ket{q}\int w_pdp.  
\end{equation}
He is not persuaded that my argument demonstrates that the last term
is negligible. He says that numerically
$\bra{q}|(\tilde{V}-\tilde{V}_{w})\ket{q} \sim 20
\bra{q}|\tilde{V}\ket{4}$ (I presume for normalized modes), and that
therefore $\int w_pdp\ll w_4$ is not sufficient criterion.

Clarifying my argument, I claim that
$\bra{q}|\tilde V \ket{4}\sim 16\omega$ and that, if $\int w_q$ is
determined ignoring the final term, then
$\int w _q dq/w_4 ={-i\pi\bra{q}|\tilde{V}\ket{4}\over
  q_0(1/\omega_r^2-1)} \sim{-i\pi\bra{q}|\tilde{V}\ket{4}\over
  4k_\perp/\omega_r} \sim 4\pi\omega\omega_r/k_\perp\sim 4\pi \omega$.
This is of order the small quantity $\omega$.

If it is true that $\bra{q}|(\tilde{V}-\tilde{V}_{w})\ket{q} \sim 20
\bra{q}|\tilde{V}\ket{4}$, then for the final term to be
negligible, we need $20\int w_pdp\ll w_4$, which is
$$
\int w_pdp/ w_4 \sim 4\pi\omega \ll 1/20
\quad i.e. \quad \omega\ll \sim 1/(80\pi).
$$
Admittedly that is a very small frequency! Thus the factor of 20, if
true, is a significant problem, and we ought to verify it.
\fi

\end{document}

%%% Local Variables:
%%% mode: latex
%%% TeX-master: t
%%% End:
